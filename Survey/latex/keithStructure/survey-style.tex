%----------------------------------------------------------------
%
%  File    :  survey-style.tex
%
%  Author  :  Keith Andrews, IICM, TU Graz, Austria
% 
%  Created :  27 May 93
% 
%  Changed :  19 Feb 2004
% 
%----------------------------------------------------------------


\chapter{Language and Writing Style}

\label{chap:Style}



The classic reference for English writing style and grammar is
\citet{StrunkWhite}. The original text is now available for free
online \citep{Strunk}, so there is no excuse at all for writing poor
English. Readers should consult it first, then continue reading this
chapter. Another good free guide is \citet{NASAGuide}.

\citet{Zobel-WritingCompSci} and \citet{BugsInWriting} are guides
specifically aimed at computer science students.
\citet{Phillips-HowGetPhD} gives practical advice for PhD
students.

The following Sections~\ref{sec:Clear} and \ref{sec:Gender} are
adapted from the CHI'94 language and writing style guidelines.







\section{Some Basic Rules of English}

There are a few basic rules of English for academic writing, which are
broken regularly by my students, particularly if they are non-native
speakers of English. Here are some classic and often encountered
examples:

\begin{itemize}

\item \emph{Never} use I, we, or you.

Write in the passive voice (third person).

\begin{tabular}{lp{0.9\hsize}}
Bad:  & You can do this in two ways.   \\
Good: & There are two ways this can be done.  \\
\end{tabular}



\item \emph{Never} use he or she, his or her.

Write in the passive voice (third person).

\begin{tabular}{lp{0.9\hsize}}
Bad:  & The user speaks his thoughts out loud.   \\
Good: & The thoughts of the user are spoken out loud.  \\
\end{tabular}

See Section~\ref{sec:Gender} for many more examples.



\item Stick to a consistent dialect of English. Choose either
  British or American English and keep to it throughout the
  whole of your thesis.



\item Do \emph{not} use slang abbreviations such as ``it's'',
  ``doesn't'', or ``don't''.

Write the words out in full: ``it is'', ``does not'', and ``do not''.

\begin{tabular}{lp{0.9\hsize}}
Bad:  & It's very simple to\ldots       \\
Good: & It is very simple to\ldots       \\
\end{tabular}



\item Do \emph{not} use abbreviations such as ``e.g.'' or
  ``i.e.''. 

Write the words out in full: ``for example'' and ``that is''.

\begin{tabular}{lp{0.9\hsize}}
Bad:  & \ldots in a tree, e.g. the items\ldots        \\
Good: & \ldots in a tree, for example the items\ldots   \\
\end{tabular}



\item Do \emph{not} use slang such as ``a lot of''.

\begin{tabular}{lp{0.9\hsize}}
Bad:  & There are a lot of features\ldots       \\
Good: & There are many features\ldots       \\
\end{tabular}



\item Do \emph{not} use slang such as ``OK'' or ``big''.

\begin{tabular}{lp{0.9\hsize}}
Bad:  & \ldots are represented by big areas.    \\
Good: & \ldots are represented by large areas.  \\
\end{tabular}



\item Do \emph{not} use slang such as ``gets'' or ``got''.

Use ``becomes'' or ``obtains'', or use the passive voice (third
person).

\begin{tabular}{lp{0.9\hsize}}
Bad:  & The radius gets increased\ldots       \\
Good: & The radius is increased\ldots         \\
\end{tabular}

\begin{tabular}{lp{0.9\hsize}}
Bad:  & The user gets disoriented\ldots       \\
Good: & The user becomes disoriented\ldots    \\
\end{tabular}




\item \emph{Never} start a sentence with ``But''.

Use ``However,'' or ``Nevertheless,''. Or consider joining the
sentence to the previous sentence with a comma.

\begin{tabular}{lp{0.9\hsize}}
Bad:  & But there are numerous possibilities\ldots       \\
Good: & However, there are numerous possibilities\ldots  \\
\end{tabular}



\item \emph{Never} start a sentence with ``Because''.

Use ``Since'', ``Owing to'', or ``Due to''. Or turn the two
halves of the sentence around.




\item \emph{Never} start a sentence with ``Also''. Also should
be placed in the middle of the sentence.

\begin{tabular}{lp{0.9\hsize}}
Bad:  & Also the target users are considered. \\
Good: & The target users are also considered. \\
\end{tabular}



\item Do \emph{not} use ``that'' as a connecting word.

Use ``which''.

\begin{tabular}{lp{0.9\hsize}}
Bad:  & \ldots a good solution that can be computed easily.  \\
Good: & \ldots a good solution which can be computed easily.  \\
\end{tabular}




\item Do \emph{not} write single-sentence paragraphs. 

Avoid writing two-sentence paragraphs. A paragraph should contain at
least three, if not more, sentences.


\end{itemize}



% rules on the use of a comma in lists
% http://en.wikipedia.org/wiki/Serial_comma








\section{Avoid Austrianisms}
\label{sec:Austrianisms}


I see these mistakes time and time again. Please do not
let me read one of them in your work.



\begin{itemize}


\item ``actual'' \neqsym ``current'' 

If you mean ``aktuell'' in German, you probably mean
``current'' in English.

\begin{tabular}{lp{0.9\hsize}}
Bad:  & The actual selection is cancelled. \\
Good: & The current selection is cancelled. \\
\end{tabular}



\item ``allows to'' is not English.

\begin{tabular}{lp{0.9\hsize}}
Bad:  & The prototype allows to arrange components\ldots \\
Good: & The prototype supports the arrangement of components\ldots \\
\end{tabular}

% they allow to achieve



\item ``enables to'' is not English.

\begin{tabular}{lp{0.9\hsize}}
Bad:  & it enables to recognise meanings\ldots \\
Good: & it enables the recognition of meanings\ldots \\
\end{tabular}


\item ``according'' \neqsym ``corresponding'' 

\begin{tabular}{lp{0.9\hsize}}
Bad:  & For each browser, an according package is created. \\
Good: & For each browser, a corresponding package is created. \\
\end{tabular}



\item ``per default'' is not English.

Use ``by default''.

\begin{tabular}{lp{0.9\hsize}}
Bad:  & Per default, the cursor is red. \\
Good: & By default, the cursor is red. \\
\end{tabular}




\item ``As opposed to'' is not English.

Use ``In contrast to''.

\begin{tabular}{lp{0.9\hsize}}
Bad:  & As opposed to C, Java is object-oriented. \\
Good: & In contrast to C, Java is object-oriented. \\
\end{tabular}


\item ``\emph{anything}-dimensional'' is spelt with a hyphen.

For example: two-dimensional, three-dimensional.



\item ``\emph{anything}-based'' is spelt with a hyphen.

For example: tree-based, location-based.



\item ``\emph{anything}-oriented'' is spelt with a hyphen.

For example: object-oriented, display-oriented.


\item ``\emph{anything}-side'' is spelt with a hyphen.

For example: client-side, server-side.


\item ``\emph{anything}-friendly'' is spelt with a hyphen.

For example: user-friendly, customer-friendly.


\item ``\emph{anything}-to-use'' is spelt with hyphens.

For example: hard-to-use, easy-to-use.



\item ``realtime'' is spelt with a hyphen if used as
  an adjective, or as two separate words if used as a noun.

\begin{tabular}{lp{0.9\hsize}}
Bad:  & \ldots display the object in realtime.  \\
Bad:  & \ldots using realtime shadow casting.   \\
Good: & \ldots display the object in real time. \\
Good: & \ldots using real-time shadow casting.  \\
\end{tabular}


\end{itemize}












\section{Clear Writing}
\label{sec:Clear}

The written and spoken language of your thesis is English as
appropriate for presentation to an international audience. Please take
special care to insure that your work is adapted to such an audience.
In particular:

\begin{itemize}
\item Write in a straight-forward style, using simple sentence
  structure.

\item Use common and basic vocabulary. For example, use ``unusual''
  for ``arcane'', and ``specialised'' for ``erudite''.

\item Briefly define or explain all technical vocabulary the first
  time it is mentioned, to ensure that the reader understands it.

\item Explain all acronyms and abbreviations. For example, the first
  time an acronym is used, write it out in full and place the acronym
  in parentheses.

\begin{tabular}{lp{0.9\hsize}}
Bad:  & \ldots When using the GUI version, the use may\ldots  \\
Good: & \ldots When using the Graphical User Interface (GUI) version, the use may\ldots  \\
\end{tabular}


\item Avoid local references. For example, not everyone knows the
  names of all the provincial capitals of Austria. If local context is
  important to the material, describe it fully.

\item Avoid ``insider'' comments. Ensure that your whole audience
  understands any reference whose meaning you do not describe. For
  example, do not assume that everyone has used a Macintosh or a
  particular application.

\item Do not ``play on words''. For example, do not use ``puns'',
  particularly in the title of a piece. Phrases such as ``red
  herring'' require cultural as well as technical knowledge of
  English.

\item Use unambiguous formats to represent culturally localised things
  such as times, dates, personal names, currencies, and even
  numbers. 9/11 is the 9th of November in most of the world.

\item Be careful with humour. In particular, irony and sarcasm can be
  hard to detect if you are not a native speaker.

\item If you find yourself repeating the same word or phrase too often,
  look in a thesaurus such as \citet{Roget,RogetInt} for an
  alternative word with the same meaning.
\end{itemize}


Clear writing experts recognise that part of writing understandable
documents is understanding and responding to the needs of the intended
audience. It is the writer's job to maintain the audience's
willingness to go on reading the document. Readers who are continually
stumped by long words or offended by a pompous tone are likely to stop
reading and miss the intended message.








\section{Avoiding Gender Bias}
\label{sec:Gender}

Part of striking the right tone is handling gender-linked terms
sensitively. Use of gender terms is controversial. Some writers use
the generic masculine exclusively, but this offends many readers.
Other writers are experimenting with ways to make English more
neutral. Avoiding gender bias in writing involves two kinds of
sensitivity:
\begin{enumerate}
\item being aware of potential bias in the kinds of observations and
  characterisations that it is appropriate to make about women and men,
  and

\item being aware of certain biases that are inherent in the language
  and of how you can avoid them.
\end{enumerate}


The second category includes using gender-specific nouns and pronouns
appropriately. Here are some guidelines for handling these
problems:
\begin{itemize}

\item Use a gender-neutral term when speaking generically of people:

\begin{tabular}{ll}
   man                 &   the human race        \\
   mankind             &   humankind, people     \\
   manpower            &   workforce, personnel  \\
   man on the street   &   average person        \\
\end{tabular}


\item Avoid clearly gender-marked titles. Use neutral terms when
good ones are available. For example:

\begin{tabular}{ll}
  chairman     &  chairperson               \\
  spokesman    &  speaker, representative   \\
  policeman    &  police officer            \\
  stewardess   &  flight attendant          \\
\end{tabular}



\item If you are speaking of the holder of a position and you know the
  gender of the person who currently occupies the position, use the
  appropriate gender pronoun.  For example, suppose the ``head nurse''
  is a man:

\begin{tabular}{lp{0.9\hsize}}
Bad:  & The head nurse must file her report every Tuesday. \\
Good: & The head nurse must file his report every Tuesday. \\
\end{tabular}



\item Rewrite sentences to avoid using gender pronouns. For example,
  use the appropriate title or job name again:

\begin{tabular}{lp{0.9\hsize}}
Bad: &
Interview the user first and then ask him to fill out a questionnaire. \\
%
Good: &
Interview the user first and then ask the user to fill out a questionnaire. \\
\end{tabular}



\item To avoid using the third person singular pronoun (his or her),
  recast your statement in the plural:

\begin{tabular}{lp{0.9\hsize}}
Bad:  & Each student should bring his text to class. \\
Good: & All students should bring their texts to class. \\
\end{tabular}



\item Address your readers directly in the second person, if it is
  appropriate to do so:

\begin{tabular}{lp{0.9\hsize}}
Bad:  & The student must send in his application by the final deadline date. \\
Good: & Send in your application by the final deadline date. \\
\end{tabular}



\item Replace third person singular possessives with articles.

\begin{tabular}{lp{0.9\hsize}}
Bad:  & Every student must hand his report in on Friday. \\
Good: & Every student must hand the report in on Friday. \\
\end{tabular}



\item Write your way out of the problem by using the passive voice.

\begin{tabular}{lp{0.9\hsize}}
Bad:  & Each department head should do his own projections. \\
Good: & Projections should be done by each department head. \\
\end{tabular}



\item Avoid writing awkward formulations such as ``s/he,'' ``he/she,''
  or ``his/her.''  They interfere when someone is trying to read a
  text aloud.  If none of the other guidelines has been helpful, use
  the slightly less awkward forms ``he or she,'' and ``his or hers.''

\end{itemize}
Remember, the goal is to avoid constructions that will offend your
readers so much as to distract them from the content of your work.




\section{When to use Capitalisation}

\emph{Capitalisation} means using a capital (upper case) initial
letter for a word. \emph{Lowercasing} means writing the entire word in
lower case. In many common writing styles, headings and titles are
partially capitalised: the first and the principal (main) words are
capitalised and other words are lowercased.

Proper names, such as the names of people, towns, and countries, are
always capitalised (Keith Andrews, the United Kingdom).



\subsection{Titles and Headings}

Capitalise all principal words: nouns, pronouns, adjectives, verbs,
and adverbs, and the first word. Lowercase all articles, coordinating
conjunctions (``for'', ``and'', ``nor'', ``but'', ``or'', ``yet'',
``so''), and prepositions.

For example:
\begin{itemize}

\item Here, ``it'' is a pronoun, which should always be capitalised.

\begin{tabular}{lp{0.9\hsize}}
Bad:  & Saying it Directly \\
Good: & Saying It Directly \\
\end{tabular}



\item Here, ``is'' is a verb, which should always be capitalised.

\begin{tabular}{lp{0.9\hsize}}
Bad:  & When is Enough Enough? \\
Good: & When Is Enough Enough? \\
\end{tabular}



\item Here, ``in'' is being used as a preposition and should be 
lowercased.

\begin{tabular}{lp{0.9\hsize}}
Bad:  & The Elephant In the Room. \\
Good: & The Elephant in the Room. \\
\end{tabular}


\item Here, ``in'' is being used as an adverb and should be 
capitalised.

\begin{tabular}{lp{0.9\hsize}}
Bad:  & Handing in Your Work. \\
Good: & Handing In Your Work. \\
\end{tabular}


\end{itemize}

See \citet{WB-Capitalisation} for some slightly different rules and
some more examples.


% Capitalization in Titles
% http://www.writersblock.ca/tips/monthtip/tipmar98.htm




\subsection{Captions}

The short version of a caption for a figure, table, or listing should
be capitalised like a heading. It appears in the List of Figures or
List of Tables. The long version of a caption for a figure, table, or
listing should be written as full sentences, i.e. only the first word
of each sentence and any proper names are capitalised.



\subsection{Chapters, Sections, Figures and Tables}

A specific, named or numbered entity, such as a particular chapter,
section, figure, table, or listing is considered to be a proper name
and thus the word Chapter, Section, Figure, etc.\ \emph{should be
  capitalised}. For example, Chapter~\ref{chap:Browsers},
Section~\ref{sec:Gender}, Figure~\ref{fig:Pistol},
Table~\ref{tab:WinIconicLang}, or Listing~\ref{list:BibFile}. However,
if an entity is not specifically named or numbered, then it should
\emph{not} be capitalised. For example, when refering to the first
chapter or the next section, without giving a name or number.









\section{Use a Spelling Checker}

In these days of high technology, spelling mistakes and typos are
inexcusable. It is \emph{very} irritating for your supervisor to have
to read through and correct spelling mistake after spelling mistake
which could have been caught by an automated spelling checker.
Believe me, irritating your supervisor is not a good idea.

So, use a spelling checker \emph{before} you hand in \emph{any}
version, whether it is a draft or a final version.
Since this is apparently often forgotten, and sometimes even wilfully
ignored, let me make it absolutely clear:
\begin{quote}
\begin{em}
Use a spelling checker, please. \\
Use a spelling checker! \\
Use a spelling checker, you moron. \\
\end{em}
\end{quote}





\section{Use a Dictionary}

If you are not quite sure of the meaning of a word, then use a
dictionary.  \citet{DictionaryCom} is a free English dictionary,
\citet{DictChemnitz} and \citet{DictLeoOrg} are two very good
English-German dictionaries.




\section{Use a Thesaurus}

If a word has been used several times already, and using another
equivalent word might improve the readability of the text, then
consult a thesaurus. \citet{Roget} and \citet{RogetInt} are free
English thesauri.


