%----------------------------------------------------------------
%
%  File    :  survey-academic.tex
%
%  Author  :  Keith Andrews, IICM, TU Graz, Austria
% 
%  Created :  27 May 93
% 
%  Changed :  22 Oct 2012
% 
%----------------------------------------------------------------


\chapter{Academic Writing}

\label{chap:Academic}




Writing in an academic context is different to other types of
writing. Care must be taken to follow the conventions of academic
writing.



\section{Academic Criteria}

An academic survey must demonstrate the following components:
\begin{itemize}
\item Motivation. What problem you are addressing and why.

\item Survey. A thorough review of related work in the field.

\item An extensive bibliography. To demonstrate knowledge of the major
  works in the field, even if they have not all been read in their
  entirety.
\end{itemize}






\section{Academic Integrity}

It is very easy to find helpful material on the web. Resist the
temptation to copy such material verbatim, even with minor changes in
phrasing and word order. It is just as easy for a supervisor or
advisor (or anyone else for that matter) to check the originality of a
piece of text by copying a passage into Google or services such as
\citep{PlagiarismOrg}.

Work submitted for academic assessment must be original and created by
the stated author(s). Care must be taken to avoid both
\emph{plagiarism} and \emph{breach of copyright}:
\begin{itemize}
\item \liintro{Plagiarism}: Using the work of others \emph{without
  acknowledgement}.

\item \liintro{Breach of copyright}: Using the work of others
  \emph{without permission}.
\end{itemize}






\subsection{Plagiarism}

Plagiarism is a violation of intellectual honesty. This means copying
other people's work or ideas without due acknowledgement, thus giving
the reader the impression that these are original (your own) work and
ideas. The Concise Oxford Dictionary, 8th Edition, defines plagiarism
as:
\begin{quotation}
\noindent
``\textbf{plagiarise}
\textbf{1} take and use (the thoughts, writings, inventions, etc.\ of
another person) as one's own. \textbf{2} pass off the thoughts etc.\
of (another person) as one's own.''
\end{quotation}
Plagiarism is the most serious violation of academic integrity and can
have dire consequences, including suspension and expulsion
\citep{Reisman2005}.



\subsection{Breach of Copyright}

Copyright law\footnote{Disclaimer: I am not a lawyer. The comments
  here reflect the situation to the best of my knowledge at the time
  of writing, but do not constitute legal advice. Laws sometimes
  change and I make no guarantees.} varies in detail from country to
country, but certain aspects are internationally widely accepted. In
general, the creator of a work, say a piece of writing, a diagram, a
photograph, or a screenshot, automatically has copyright of that
work. Copyright usually expires 70 years after the creator's
death. The copyright holder can grant the right for others to use or
publish their work on an exclusive or non-exclusive basis.

The copyright laws of most countries have provisions for \emph{fair
  use}, which generally means it is allowable to quote small parts of
a work. Austrian copyright law \citep[§ 42f]{UrhG} allows for
reasonable quotes from published works or works made publicly
available with permission of the copyright holder. Austrian copyright
law \citep[§ 42g]{UrhG} also makes certain exemptions for materials
used for teaching in universities and schools. Note that significant
changes to Austrian copyright law \citep{UrhG-Novelle-2015} came into
effect on \yearmonthday{2015}{10}{1}.





\section{Acceptable Use}

Academic work almost always builds upon the work of others, and it is
appropriate, indeed essential, that related and previous work by
others be discussed in an academic thesis. However, this must be done
according to the rules of acceptable use. There are two forms of
acceptable use:
\begin{itemize}
\item \liintro{Paraphrasing (Indirect Quotation)}: Summarising the ideas
  of someone else using original words and with attribution.
\item \liintro{Quoting (Direct Quotation)}: Including an exact
  verbatim copy inside quotation marks and with attribution.
\end{itemize}
Attribution means that the original source is cited.
For further information on acceptable and non-acceptable academic
practice see \citep{FremdeFedern,Wikipedia-Zitat}.




\subsection{Paraphrasing (Indirect Quotation)}

Paraphrasing means closely summarising and restating the ideas of
another person, but in (your own) original words. When writing a
literature survey, the relevant parts of each paper or source are
generally \emph{paraphrasd}.

One good technique for paraphrasing is:
\begin{enumerate}
\item Read the original source.
\item Put it down away from view.
\item \emph{Without refering to the original}, summarise it in your own words.
\end{enumerate}
When paraphrasing someone else's ideas, the original source must
always be cited!

Since paraphrased text is not enclosed in quotation marks, it is not
always obvious how to indicate the extent of the text which
corresponds to a particular citation. If the paraphrased text only
covers a single paragraph, include the citation either within or at
the end of the first sentence of the paragraph, or at the end of the
paragraph. Otherwise, describe the extent of the citation in words at
the beginning, for example: This section is based on the work of
\citet{InfoSkyIVS}.




\subsection{Quoting Text (Direct Quotation)}

In some circumstances, it makes sense to directly \emph{quote} small
parts of text (typically a few sentences or paragraphs) from a
relevant source. When quoting directly, the \emph{exact} words,
spelling, and punctuation of the original are copied verbatim and
enclosed in quotation marks.

Most of an academic paper or thesis must be in words written by the
author(s) themselves. However, when an exact phrase or specific
wording from another source is important, then a direct quotation
should be used. In any case, the original source must be cited!




\subsection{Quoting Images}

It is common to want to include photographs, diagrams, or screenshots
taken from the internet or from another work, particularly when
surveying related work. Austrian copyright law \citep[§ 42f]{UrhG}
allows images from published works to be included in scientific works
for the purposes of discussion. However, it is not entirely clear what
constitutes a scientific work and what not. The safest policy is
always to ask permission from the owner.

For diagrams, an alternative strategy is to redraw and possibly adapt
the diagram in a drawing editor such as Adobe Illustrator
\citep{Adobe-Illustrator} or Inkscape \citep{Inkscape}. The original
source should be cited with wording like ``Redrawn from [\ldots]'' or
``Adapted from [\ldots]''.

For graphs and plots, it is often possible to reconstruct the graphic
from the original data using tools such as gnuplot \citep{gnuplot} or
R \citep{R-Project}. The original source should be cited with wording
similar to ``Created from the original data [\ldots] using[\ldots]''.

For screenshots, it is sometimes possible to obtain the original
software, install it, and make new screenshots. The source software
should be cited with wording similar to ``Screenshot created using
[\ldots]''.




\subsection{Always State Both Source and Permission}

Regardless of whether permission has been obtained from the copyright
owner or material is being used under the provisions of a specific
country's copyright law: whenever someone else's work is being used,
academic integrity dictates that the original source must be cited!
In addition, it is also good practice to state the terms (permission)
under which the material is being used.


For each piece of included material, make two things absolutly clear:
\begin{enumerate}
\item \liintro{Source}: Cite the original source of the material.
  Use a standard \LaTeXe citation.

\item \liintro{Permission}: Explain the legal basis for using the
  material. For example, give the \emph{exact} Creative Commons
  licence, state the \emph{exact} legal exemption, or state that
  permission has kindly been given by the named original author.
\end{enumerate}


These two things should be stated in two places:
\begin{itemize}
\item \liintro{Caption}: At the end of the caption of the figure
  or listing.

\item \liintro{Credits section}: In the Credits section at the
  front of the thesis.
\end{itemize}


All this means, of course, that if a thesis is based upon this
skeleton \citep{KeithThesis}, then the source and permission should be
stated at the appropriate place (in this case, in the Credits
section).







\section{References}



\subsection{Bib Files}

Typically, one or more \vname{.bib} files are prepared, containing
various original sources and references.
Listing~\ref{list:BibFile} shows four typical entries from a
\vname{.bib} file for use with biblatex and biber. The
\vname{inproceedings} entry describes a paper published in conference
proceedings, the \vname{article} entry describes a paper published in
a journal, and the \vname{booklet} entry is being used for internet
resources and web sites (\vname{booklet} has the advantage over
\vname{online} that it has a \vname{howpublished} field.).



\lstinputlisting[%
  float=tp,
  xleftmargin=0cm,
  xrightmargin=0cm,
  language=biblatex,
  basicstyle=\footnotesize\ttfamily,
  frame=shadowbox,
  numbers=left,
  label=list:BibFile,
  caption={[Four Typical Entries from a \vname{.bib} File]%
Four typical entries from a \vname{.bib} file for use
with biblatex and biber.
An \vname{inproceedings} entry describes a paper published
in conference proceedings, an \vname{article} entry describes
a paper published in a journal, and a \vname{booklet} entry
is used for internet resources and web sites.
The \vname{doi} field gives
the DOI (digital object identifier) of the paper.},
]
{listings/some.bib}


Of particular note is the \vname{doi} field, which gives the DOI
(digital object identifier) of a paper. DOIs are for academic papers
what ISBNs are for books; a unique handle with which one can easily
find the original. Most publishers are now assigning DOIs to new
conference and journal papers and are working back in time to assign
them to previously published papers. Always give the DOI of a paper
where one is available. If a DOI exists but points to a subscription
site, and the paper is also freely available on the web (say at the
home page of an author), then use the \vname{url} field to give the
free URL as well. Do not redundantly give the same URL in the
\vname{url} field which the DOI itself resolves to.





\subsection{Downloading Bib Entries}

\begin{lstlisting}[%
  float=tp,
  xleftmargin=0cm,
  xrightmargin=0cm,
  language=biblatex,
  basicstyle=\footnotesize\ttfamily,
  frame=shadowbox,
  numbers=left,
  label=list:BibACMIEEE,
  caption={[Massaging Bib Entries from ACM and IEEE]%
Bib entries copied from the ACM Digital Library or the
IEEE Computer Society Digital Library contain useful information,
but cannot be used ``as-is''. They must be edited to conform
to biblatex and to these thesis guidelines.
},
]
% From the IEEE Computer Society DL:

@article{10.1109/INFOVIS.2005.7,
author = {Martin Wattenberg},
title = {Baby Names, Visualization, and Social Data Analysis},
journal = {infovis},
volume = {0},
year = {2005},
issn = {1522-404x},
pages = {1},
doi = {http://doi.ieeecomputersociety.org/10.1109/INFOVIS.2005.7},
publisher = {IEEE Computer Society},
address = {Los Alamitos, CA, USA},
}


% From the ACM DL:

@inproceedings{1106568,
 author = {Martin Wattenberg},
 title = {Baby Names, Visualization, and Social Data Analysis},
 booktitle = {INFOVIS '05: Proceedings of the Proceedings of the 2005 IEEE Symposium on Information Visualization},
 year = {2005},
 isbn = {0-7803-9464-x},
 pages = {1},
 doi = {http://dx.doi.org/10.1109/INFOVIS.2005.7},
 publisher = {IEEE Computer Society},
 address = {Washington, DC, USA},
 }


% Clean, edited version for Keith:

@inproceedings{WattenbergNames,
  author       = "Martin Wattenberg",
  title        = "Baby Names, Visualization, and Social Data Analysis",
  booktitle    = "Proc.\ {IEEE} Symposium on Information Visualization
                  (InfoVis 2005)",
  location     = "Minneapolis, Minnesota, USA",
  organization = "{IEEE} Computer Society",
  isbn         = "078039464X",
  date         = "2005-10",
  pages        = "1--8",
  doi          = "10.1109/INFOVIS.2005.7",
  url          = "http://www.research.ibm.com/visual/papers/final-baby-margin-nocomments.pdf",
}
\end{lstlisting}



When \vname{.bib} entries are downloaded or copied from the ACM
Digital Library, the IEEE Digital Library, or other onlibne sources,
they should \emph{not} be used as is. They generally need to be
cleaned up first and made consistent with biblatex.
Listing~\ref{list:BibACMIEEE} shows typical BibTeX entries provided by
the ACM Digital Library and the IEEE Computer Society Digital Library.


To bring bib entries into line with biblatex and the examples shown in
Listing~\ref{list:BibFile}, the following should be addressed:
\begin{itemize}
\item The title of the paper should use capitalised main words.

\item Capitalisations in the title which need to be preserved (such as
  the R in VRwave) should be enclosed in curly brackets ({VRwave}).

\item The \vname{title} and \vname{booktitle} should use
  capitalised main words (not all lower case).

\item The \vname{edition} field is usually be a number in inverted
  commas, such as \verb|"2"|, instead of a word such as
  \verb|"Second"|.

\item The name of a conference should be rephrased, with the short
  form of the conference name in parentheses at the end (VRML'98).

\item Any \vname{year}, \vname{month}, and \vname{day}
  fields should be combined into a \vname{date} field.

\item For a conference paper, the first day of the conference
  should be used as the date of publication.

\item The location of a conference should be in the \vname{venue}
  field, not in the \vname{address} or \vname{location} field.
  The \vname{address} field is for the address of the publisher.


\item Any minus signs must be removed from the ISBN number.
  Otherwise, the macro used in this skeleton for handling ISBNs and
  linking to Amazon will break.

\item Any initial \vname{http://doi.acm.org/} or
  \vname{http://doi.ieeecomputersociety.org/} must be removed from
  the DOI. They are \emph{not} part of the DOI.

\item If a free, unofficial version of a paper with a DOI is available
  at the web site of one of the authors, give this in the \vname{url}
  field.

\item Manually shorten any URL as much as possible: try selectively
  removing parameters after a question mark and try removing
  \vname{www} from the domain. Do \emph{not} use a URL shortening
  service like \website{bit.ly}, since there is no guarantee the
  service will be around long term. It is acceptable to use a URL
  shortening service maintained by the original site themselves, such
  as \website{youtu.be} for YouTube URLs.

\end{itemize}






\subsection{What to Reference}

The set of references should be balanced:
\begin{itemize}
\item Do not have largely web sites as references. A few web sites as
  references is fine, most references being web sites is (usually) not
  so good.

\item Do not have too many Wikipedia references. A few Wikipedia
  references is OK; more than a few is not. Wikipedia is a good
  \emph{starting} point for (further) academic research, it is not a
  good ending point for academic research.

\item Have plenty of academic conference and journal papers (with a
  DOI). Sometimes, both an academic paper and a project web site will
  be avilable -- reference both as separate entries.

\item Include some books (with an ISBN) if at all possible. Books
  still count in academic circles.
 
\item If you know or suspect who will be reviewing or marking your
  thesis or paper, make sure to include some of their references. The
  first thing many reviewers do is check to see if they appear in the
  bibliography.

\item No ghost references. Every reference in the bibliography should
  be cited somewhere in the text.

\end{itemize}





\subsection{Citing}

When a citation is included within flowing text:
\begin{itemize}
\item Distinguish between \emph{textual} citations and
  \emph{parenthetical} citations. Textual citations are used to embed
  the authors' names in the current sentence. Parenthetical citations
  are used at the end of a sentence.
\begin{quote}
\verb|\citet{Jones1990}| \rightarrowsym Jones et al. [1990] \\
\verb|\citep{Jones1990}| \rightarrowsym [Jones et al., 1990]
\end{quote}


\item If one specific part in a very long paper or book is being
  cited, always state the page number or range in the citation:
\begin{quote}
\verb|As \citet[pages 22--23]{Jones1990} say| \rightarrowsym As Jones et al. [1990, pages 22–23] say\\
\end{quote}





\section{Guides to Scientific Writing}

\citet{CraftScientificWriting} is one of the classic guides to
scientific writing. Other good ones include \citet{BoothCraft}
and \citet{BoothCommunicating}.



\end{itemize}

