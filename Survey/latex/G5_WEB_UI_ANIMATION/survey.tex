%----------------------------------------------------------------
%
%  File    :  survey.tex
%
%  Author  :  Keith Andrews, IICM, TU Graz, Austria
%
%  Created :  24 Mar 2010
%
%  Changed :  22 Nov 2016
%
%----------------------------------------------------------------


\documentclass[11pt,onecolumn,twoside]{report}

\usepackage[          % set page and margin sizes
  a4paper,
  twoside,
  top=5mm,
  bottom=10mm,
  inner=15mm,
  outer=15mm,
  bindingoffset=10mm,
  head=10mm,
  foot=10mm,
  headsep=15mm,
  footskip=15mm,
  includeheadfoot,
]{geometry}
% A4 is 210 x 297 mm



\usepackage{txfonts}            % new times fonts
\usepackage{relsize}            % relative font sizes \smaller \larger
\usepackage{float}              % H for float placement
\usepackage{setspace}           % line spacing

\usepackage[T1]{fontenc}        % 8-bit output chars (must be before inputenx)
\usepackage[utf8]{inputenx}     % input char encoding

\usepackage{textcomp}           % symbols such as \texttimes and \texteuro
\usepackage{latexsym}

\usepackage{xspace}
\usepackage{etoolbox}           % for \newrobustcmd
\usepackage{makecmds}           % for \makecommand


\usepackage[english,austrian,british]{babel}


\usepackage[bf,sf]{titlesec}



\setlength{\textfloatsep}{10mm plus 2mm minus 1mm}
\setlength{\floatsep}{10mm plus 2mm minus 1mm}
\setlength{\intextsep}{10mm plus 2mm minus 1mm}

\setlength{\dbltextfloatsep}{10mm plus 2mm minus 1mm}
\setlength{\dblfloatsep}{10mm plus 2mm minus 1mm}

\setlength{\abovecaptionskip}{4mm plus 2mm minus 1mm}
\setlength{\belowcaptionskip}{0mm}




% use caption and subfig (caption2 and subfigure are now obsolete)

\usepackage[
  position=bottom,
  margin=1cm,
  font=small,
  labelfont={bf,sf},
  format=hang,
  indention=0mm,
]{caption,subfig}

\captionsetup[subfigure]{
  margin=0pt,
  parskip=0pt,
  hangindent=0pt,
  indention=0pt,
  singlelinecheck=true,
}




% fancyhdr to make nice headers and footers
% and deal with long chapter names

\usepackage{fancyhdr}         % headers and footers
\pagestyle{fancy}             % must call to set defaults before redefining

\renewcommand{\chaptermark}[1]{%
  \markboth{\thechapter.\ #1}{}
}
\renewcommand{\sectionmark}[1]{%
  \markright{\thesection.\ #1}
}
\renewcommand{\headrulewidth}{0mm}
\renewcommand{\footrulewidth}{0mm}
\newcommand{\headlook}{\sffamily}
\fancyhf{}
\fancyhead[LE,RO]{\thepage}
\fancyhead[LO]{%
\parbox[t]{0.8\textwidth}{\headlook\nouppercase{\rightmark}}
}
\fancyhead[RE]{%
\parbox[t]{0.8\textwidth}{\raggedleft\headlook\nouppercase{\leftmark}}
}


%\fancypagestyle{plain}{%   redefine plain style, but doesn't work
%  \fancyhf{}    % clear all header and footer fields
%  \fancyfoot[C]{\headlook \thepage} % except the center
%  \renewcommand{\headrulewidth}{0pt}
%  \renewcommand{\footrulewidth}{0pt}
%}



\usepackage{xcolor}
\definecolor{darkgreen}{rgb}{0.0,0.2,0.0}
\definecolor{darkblue}{rgb}{0.0,0.0,0.2}
\definecolor{darkred}{rgb}{0.2,0.0,0.0}
\definecolor{verylightgrey}{gray}{0.95}
\definecolor{lightgrey}{gray}{0.9}
\definecolor{black}{gray}{0.0}


\usepackage{tabularx}


\usepackage{listings}                 % for listings of source code
\usepackage{calc}                     % for calulation below

\makeatletter
\newlength{\numwidth}%
\setlength{\numwidth}{\widthof{\normalfont{\lst@numberstyle{99}}}}% Up to 2-digit (99) line numbers
\def\lst@PlaceNumber{%
  \makebox[\numwidth+1em][l]{%
    \makebox[\numwidth][r]{\normalfont\lst@numberstyle{\thelstnumber}}%
  }%
}
\makeatother

\lstset{                              % set parameters for listings
  floatplacement=tp,                  % default float placement
  numberbychapter,
  inputencoding=utf8,
  language=,                          % empty = plain text
  basicstyle=\small\ttfamily,
  tabsize=2,
  xleftmargin=1cm,
  xrightmargin=1cm,
  frame=none,
  framexleftmargin=0mm,
  rulesepcolor=\color{verylightgrey},
  numbers=none,
  numberstyle=\scriptsize,
  numbersep=2ex,
  breaklines,
  showtabs=false,
  showspaces=false,
  showstringspaces=false,
  keywordstyle=\bfseries,
  identifierstyle=,
  stringstyle=,
  captionpos=b,
  abovecaptionskip=\abovecaptionskip,
  belowcaptionskip=\belowcaptionskip,
  aboveskip=\floatsep,
  belowskip=\floatsep,
  extendedchars=true,
  literate=%
    {©}{{\textcopyright}}1
    {€}{{\texteuro}}1
    {Ö}{{\"O}}1
    {Ä}{{\"A}}1
    {Ü}{{\"U}}1
    {ß}{{\ss}}1
    {ö}{{\"o}}1
    {ä}{{\"a}}1
    {ü}{{\"u}}1,       % map some utf8 chars to replacements
}


\lstdefinelanguage{biblatex}   % based on biblatex v 2.7a from 2013-07-14
{
  keywords={%
    @article,@book,@mvbook,@inbook,@bookinbook,@suppbook,%
    @booklet,@collection,@mvcollection,@incollection,@suppcollection,%
    @manual,@misc,@online,@patent,@periodical,@suppperiodical,%
    @proceedings,@mvproceedings,@inproceedings,@reference,@mvreference,%
    @inreference,@report,@set,@thesis,@unpublished,@xdata,%
    @conference,@electronic,@mastersthesis,@phdthesis,@techreport,@www,%
    @artwork,@audio,@bibnote,@commentary,@image,@jurisdiction,@legislation,%
    @legal,@letter,@movie,@music,@performance,@review,@software,%
    @standard,@video%
  },
  comment=[l][\itshape]{@comment},
  sensitive=false,
}


\usepackage[short]{datetime}   % load datetime *after* babel, requires fmtcount
% \newdateformat{britdate}{%
% \ordinaldate{\THEDAY} \,\monthname[\THEMONTH] \THEYEAR
% }
\newdateformat{keithdate}{%
\twodigit{\THEDAY}~\shortmonthname[\THEMONTH]~\THEYEAR
}


\usepackage[hyphens,obeyspaces]{url}
\def\UrlFont{\smaller\ttfamily}



\usepackage[
  autostyle,
  english=british,
  threshold=0,
  thresholdtype=lines,
]{csquotes}
\renewcommand{\mkcitation}[1]{\space#1}

\newenvironment*{smallquote}          % smaller text within a block quote
  {\quote\smaller}
  {\endquote}
\SetBlockEnvironment{smallquote}

% put quotation marks around block quotes
% \renewenvironment{quoteblock}{\openautoquote}{\closeautoquote}

% I prefer double quotes as outer
\DeclareQuoteStyle[keithbritish]{british}%  [variant]{style}
  {\textquotedblleft}%                      opening outer mark
  {\textquotedblright}%                     closing outer mark
  [0.05em]%
  {\textquoteleft}%                         opening inner mark
  {\textquoteright}%                        closing inner mark

\setquotestyle[keithbritish]{british}



\usepackage[
  backend=biber,
  bibstyle=authoryear-ka,
  citestyle=authoryear-ka,
  sorting=nyt,
  useprefix,                   % van and von are part of second name
  mergedate=false,             % only for authoryear style
  dashed=false,                % only for authoryear style
  abbreviate=false,
  maxcitenames=2,              % if more than two authors, then use et al
  mincitenames=1,              % if exceeds 2 authors, then use 2
  maxbibnames=99,              % print all authors in biblio
  uniquename=init,
  hyperref=true,
  backref=true,
  backrefstyle=two,
  natbib=true,
  sortlocale=en,
]{biblatex}



% set for csquotes, but \autocite only available
% after biblatex is loaded
\SetCiteCommand{\autocite}    % or maybe \parencite

% more space between entries in bib
\setlength\bibitemsep{1.5\itemsep}


% remove URL: from in front of URLs
\DeclareFieldFormat{url}{\url{#1}}
\DeclareFieldFormat{doi}{\doi{#1}}
\DeclareFieldFormat{isbn}{\isbn{#1}}
\DeclareFieldFormat{issn}{\issn{#1}}

% suppress urldate field
\DeclareSourcemap{
  \maps[datatype=bibtex]{
    \map{
      \step[fieldset=urldate, null]
    }
  }
}

% for article titles
\DeclareFieldFormat{title:article}{\emph{#1}\midsentence}

\DefineBibliographyStrings{british}{%
  january          = {Jan},
  february         = {Feb},
  march            = {Mar},
  april            = {Apr},
  may              = {May},
  june             = {Jun},
  july             = {Jul},
  august           = {Aug},
  september        = {Sep},
  october          = {Oct},
  november         = {Nov},
  december         = {Dec},
}



% \bibliography{kandrews,latex,writing,inm-plag}

%\addbibresource{writing.bib}
%\addbibresource{latex.bib}
%\addbibresource{kandrews.bib}
%\addbibresource{ivis.bib}
\addbibresource{G5.bib}
\addbibresource{images/imgLocations.bib}




\usepackage{ifpdf}

\ifpdf
  % pdflatex
  \usepackage[pdftex]{graphicx}
  \DeclareGraphicsExtensions{.pdf,.jpg,.png}
  \pdfcompresslevel=9
  \pdfpageheight=297mm
  \pdfpagewidth=210mm
  \usepackage[         % hyperref should be last package loaded
    unicode,
    pdftex,
    pdftitle={Web UI Animation},
    pdfsubject={},
    pdfauthor={Rok Kogovšek, Alexei Kruglov, Fernando Pulido Ruiz, and Helmut Zöhrer},
    pdfkeywords={survez, IAWEB, animation, UI, web},
    bookmarks,
    bookmarksnumbered,
    linktocpage,
    colorlinks,
    linkcolor=darkred,
    anchorcolor=red,
    citecolor=darkgreen,
    urlcolor=darkblue,
    pdfview={FitH},
    pdfstartview={Fit},
    pdfpagemode=UseOutlines,       % open bookmarks in Acrobat
    plainpages=false,              % avoids duplicate page number problem
    pdfpagelabels,                 % avoids duplicate page number problem
    breaklinks=true,               % allow links exceeding a single line
  ]{hyperref}

\else
  % latex
  % should never have to run latex, since l2h now understands pdflatex .aux
  \usepackage[dvips]{graphicx}
  \usepackage[dvips]{hyperref}
  \DeclareGraphicsExtensions{.eps}
\fi





% \liintro list item intro is a style used when list items have an
% introduction phrase (say in italics) followed by a colon.
\newcommand{\liintro}[1]{\emph{#1}}


\newcommand{\imgcredit}[1]
{\smaller{}[#1]}



\newcommand{\copyrightACM}
{%
Copyright \copyright\ by the Association for Computing Machinery, Inc.%
}




\newcommand{\daymonthyear}[3]
{%
\twodigit{#1}\hspace{0.7ex}\nolinebreak[2]\shortmonthname[#2]\hspace{0.7ex}\nolinebreak[2]#3%
}


\newcommand{\monthyear}[2]
{%
\shortmonthname[#1]\hspace{0.7ex}\nolinebreak[2]#2%
}


\newcommand{\yearmonthday}[3]
{%
\twodigit{#3}\hspace{0.7ex}\nolinebreak[2]\shortmonthname[#2]\hspace{0.7ex}\nolinebreak[2]#1%
}


\newcommand{\yearmonth}[2]
{%
\shortmonthname[#2]\hspace{0.7ex}\nolinebreak[2]#1%
}



% link to Amazon or
% http://worldcatlibraries.org/wcpa/isbn/[ISBN number]

\newrobustcmd{\isbn}[1]
{%
{%
\ifpdf
{\smaller ISBN}
\href{http://www.amazon.com/exec/obidos/ASIN/#1/keithandrewshcic}{#1}%
\else
{\smaller ISBN}
#1%
\fi
}%
}



% ISSN
% http://www.bl.uk/services/bibliographic/issn.html
% 8 digits, should be printed xxxx-xxxx
% e.g. 0020-0190 is Information Processing Letters, Elsevier
%
% Lookup services:
% http://kmittlib.lib.kmutt.ac.th:81/search/i?SEARCH=0020-0190
% http://worldcatlibraries.org/wcpa/issn/0020-0190

\newrobustcmd{\issn}[1]
{%
{%
\ifpdf
{\smaller ISSN}
\href{http://worldcatlibraries.org/wcpa/issn/#1}{#1}%
\else
{\smaller ISSN}
#1%
\fi
}%
}



% DOIs  http://www.doi.org/  e.g.
% doi:10.1038/nature723
% http://dx.doi.org/10.1038/nature723

\newrobustcmd{\doi}[1]
{%
{%
\def\UrlFont{\rmfamily}
\ifpdf                                   % pdflatex
\href{http://dx.doi.org/#1}{doi:\protect\nolinkurl{#1}}%
\else                                    % latex
doi:\protect\nolinkurl{#1}%
\fi
}%
}





\newrobustcmd{\website}[1]
{%
\ifpdf                                  % pdflatex
\href{http://#1/}{\nolinkurl{#1}}%
\else                                   % latex
\nolinkurl{#1}%
\fi
}




\newcommand{\news}[1]
{%
\ifpdf
\href{news:#1}{\nolinkurl{#1}}
\else
\nolinkurl{#1}%
\fi
}








% based on url package
% define styles for class, file, and variable names
% which break nicely at line breaks

% make the macros robust so they work inside captions, etc

\newcommand{\ttname}{\begingroup \smaller\urlstyle{tt}\Url}
\newcommand{\rmname}{\begingroup \smaller\urlstyle{rm}\Url}
\newcommand{\sfname}{\begingroup \smaller\urlstyle{sf}\Url}


% cname is for class names
\newrobustcmd{\cname}[1]{\sfname{#1}}

% fname is for file names and directory names
\newrobustcmd{\fname}[1]{\ttname{#1}}

% vname is for variable names, domain names, email addresses
\newrobustcmd{\vname}[1]{\ttname{#1}}



% Euro symbol
\newcommand{\euro}{\texteuro\,}

% times symbol
\newcommand{\timessym}{\texttimes\,}

% approx symbol
\newcommand{\approxsym}{\ensuremath\approx\,}

% plusminus symbol
\newcommand{\plusminussym}{\textpm\,}

% not equal symbol
\newcommand{\neqsym}{\ensuremath{\neq\,}}

% rightarrow symbol
\newcommand{\rightarrowsym}{\ensuremath\rightarrow\,\,}




\newcommand{\TODO}[1]
{
{\textcolor{red}{[TODO: #1]}}
}



\newcommand{\fullh}{18cm}         % height of figures for 1 per page
\newcommand{\halfh}{9.5cm}        % height of figures for 2 per page
\newcommand{\thirdh}{6cm}         % height of figures for 3 per page


\tolerance=400 
  % makes some lines with lots of white space, but      
  % tends to prevent words from sticking out in the margin





\definecolor{lightgray}{rgb}{.9,.9,.9}
\definecolor{darkgray}{rgb}{.4,.4,.4}
\definecolor{purple}{rgb}{0.65, 0.12, 0.82}
\lstdefinelanguage{JavaScript}{
	keywords={break, case, catch, continue, debugger, default, delete, do, else, false, finally, for, function, if, in, instanceof, new, null, return, switch, this, throw, true, try, typeof, var, void, while, with},
	morecomment=[l]{//},
	morecomment=[s]{/*}{*/},
	morestring=[b]',
	morestring=[b]",
	ndkeywords={class, export, boolean, throw, implements, import, this},
	keywordstyle=\color{blue}\bfseries,
	ndkeywordstyle=\color{darkgray}\bfseries,
	identifierstyle=\color{black},
	commentstyle=\color{purple}\ttfamily,
	stringstyle=\color{red}\ttfamily,
	sensitive=true
}






\lstset{
	backgroundcolor=\color{lightgray},
	float=tp,
	xleftmargin=1cm,
	xrightmargin=1cm,
    framexleftmargin=1mm,
	extendedchars=true,
	basicstyle=\footnotesize\ttfamily,
	showstringspaces=false,
	showspaces=false,
	numbers=left,
	numberstyle=\footnotesize,
	numbersep=9pt,
	tabsize=2,
	breaklines=true,
	showtabs=false,
	captionpos=b
}



\begin{document}

\keithdate

\normalsize
\pagestyle{empty}         % for preliminary pages (no numbers shown)
\pagenumbering{Roman}     % for pdf labels




\begin{titlepage}

\begin{center}
{\Large \sffamily \bfseries Web UI Animation}

\vspace{1cm}

% {\large\sffamily Keith Andrews}

{\large\sffamily Group 5}

\vspace{5mm}

{\large\sffamily Rok Kogovšek, Alexei Kruglov, Fernando Pulido Ruiz, and Helmut Zöhrer}

\vspace{1cm}

% Institute for Information Systems and Computer Media (IICM), \\
% Graz University of Technology \\
% A-8010 Graz, Austria \\[1cm]


{\large
706.041 Information Architecture and Web Usability WS 2016 \\
Graz University of Technology \\
A-8010 Graz, Austria  \\[1cm]
}

\vspace{1cm}

% {\large 22 Nov 2016}

{\large \today}


\end{center}



\vspace{2cm}

\begin{quote}
\begin{center}
{\large\sffamily\bfseries Abstract}
\end{center}

TO DO

Writing a survey can be a traumatic endevour. It might be a student's
first foray into academic research. There are often obstacles and
false dawns along the way. This survey paper takes a fresh look at the
process and addresses new ways of accomplishing this daunting goal.

The abstract should concisely describe what the survey is about.
State the areas which are covered and also those which are not
covered. Market your survey to your readership. Also, make sure you
mention all relevant keywords in the abstract, since many readers read
\emph{only} the abstract and many search engines index \emph{only} the
title and the abstract.

This survey explores the issues concerning the writing of an academic
survey paper and presents numerous novel insights. Special attention
is paid to the use of clear and simple English for an international
audience, and advice is given as to the use of technical aids to
production.
\end{quote}

\vfill

\begin{center}
{\small\sffamily \copyright ~ Copyright 2016 by the author(s),
except as otherwise noted.}

\vspace{2mm}
{\footnotesize\sffamily This work is placed under a
Creative Commons Attribution 4.0 International
(\href{https://creativecommons.org/licenses/by/4.0/}{CC BY 4.0}) licence. It uses the LaTex template from "Writing a Survey Paper" by Keith Andrews, used under CC BY 4.0 / Desaturated from original
}
\end{center}

\end{titlepage}




\cleardoublepage
\pagestyle{plain}
\pagenumbering{roman}



{
\setlength{\parskip}{3pt plus 3pt minus 3pt}     % compact tables of contents
\tableofcontents
\addcontentsline{toc}{chapter}{Contents}

\cleardoublepage
\listoffigures
\addcontentsline{toc}{chapter}{List of Figures}

\cleardoublepage
\listoftables
\addcontentsline{toc}{chapter}{List of Tables}

\cleardoublepage
\renewcommand{\lstlistlistingname}{List of Listings}
\lstlistoflistings
\addcontentsline{toc}{chapter}{List of Listings}
}


\cleardoublepage
\pagestyle{headings}        % for main pages
\pagenumbering{arabic}

\cleardoublepage
%----------------------------------------------------------------
%
%  File    :  survey-intro.tex
%
%  Author  :  Keith Andrews, IICM, TU Graz, Austria
% 
%  Created :  27 May 1993
% 
%  Changed :  06 Dec 2016
% 
%----------------------------------------------------------------


\chapter{Introduction}

\label{chap:Intro}

Web animations are not only there to look pretty. Animated objects may carry meaning and we will see that there is much more thinking and planning necessary than expected at first sight. 

Knowing this makes it difficult to find the right amount of animation without overdoing it. Some animation just for the purpose that something happens on the screen will not do any good to the user and might lead to distraction. 

The basic way of implementing animations is with Cascading Style Sheets (CSS). It is widely supported and offers rather simple but effective ways of transforming objects in a meaningful way. Especially keyframes and easing effects are widely used for animating objects. 
The explicit implementation of hamburger menus, loading icons and navigation bars is described in detail in this survey. 

Similar to CSS, SVG animations are possible as well. They have the feature of including basic graphical elements and have the advantage of being infinitely scalable which might come handy in the online use. 

Another way of implementing animations is via JavaScript (JS). It allows more complex creations and gives the developer much more possibilities and freedom. It requires longer loading times and generally takes up more memory than pure CSS animations though. A combination of CSS and JS can be useful for a wide variety of animation effects. 
Especially the possibilities of animating text with JS and CSS are underlined in this survey. 


% Fernando Pulido Ruiz
\cleardoublepage
%----------------------------------------------------------------
%
%  File    :  survey-animation.tex
%
%  Author  :  Fernando Pulido Ruiz, TU Graz, Austria
% 
%  Created :  01 Dec 2016
% 
%  Changed :  X Dec 2016
% 
%----------------------------------------------------------------


\chapter{Animation}

\label{chap:Animation}

{\em"Animation is defined as changing some property over time. On the other hand, motion is the act of moving or the process of being moved. . . .  To put it more simply, all motion is animation, but not all animation is motion."}\citep{head2016designing}

\TODO{\section{sections by F.P.R.}} % (fold)
\label{sec:anime_sthg}


\begin{figure}[tp]
\centering
\includegraphics[keepaspectratio,width=\hsize,height=\halfh]
{images/storyboard.jpeg}

\caption[Storyboard Sketching]{
Example of storyboard sketching for drag and drop animation \citep{microsoftStoryboard}.
\imgcredit{Used with permission from Microsoft - Microsoft Copyrighted Content Guidelines}
}
\label{fig:storyboard}
\end{figure}

% section CSS_Examples (end)

% Rok Kogovšek
\cleardoublepage
%----------------------------------------------------------------
%
%  File    :  survey-CSS.tex
%
%  Author  :  Rok Kogovšek, TU Graz, Austria
% 
%  Created :  01 Dec 2016
% 
%  Changed :  X Dec 2016
% 
%----------------------------------------------------------------


\chapter{Cascading Style Sheets (CSS)}

\label{chap:CSS}

Knowing the usefulness of animation in web UI and the correct way of animation planing are just the fundemantals for our conceptual plans. Those still need to be implemented to get the end product and here we usually hit a wall build from the various tools, that say they can all solve our problems. Even well established people in the field have stories as such to tell. Val Head actually started with animation due to an interesting Flash workshop. Flash was at that time the de facto king in its era, however as we know, that era is already dead. Nowdays we can acomplish
all we could with Flash and more with just the core parts of the web, namely HTML, CSS and JS
\citep{head2016designing}.

\section{Do Everything You Can With CSS}

\label{sec:everythingCSS}

With Responsive web design (RWD) in our websites and animation being part of the design, see section \ref{sec:anime_dev}, it should be natural to use the guidelines of RWD also in animation planning. \citet{IAWEB} teaches us that one of the RWD guidelines is also Progressive enhancement, which is best described with words of the conceptual authors \citet{champeon2003inclusive}: {\em"Leave no one behind. . . .  accessibility is for everyone, not just the disabled"}. With CSS nowdays being a core part of the web and at the same time being the lowest web component that enables animation with RWD guidelines\footnote{Of course one can just use an animated image, e.q. a GIF with an image sequence, and just append it with HTML into the design. However, this image will become a static component of the design and will not follow RWD guidelines.}, one should always implement with CSS and HTML alone as much of the desired animation as possible. One has only to make sure the browser support for the animated attribute.

Other supporting arguments for use of CSS as the starting point for web UI animation beside responsiveness can be summarized with the the so called "Simple CSS Truths", a list of truths by \citet{palermoCSS} enhanced with the teachings of \citet{IAWEB}:

\begin{description}
\item [CSS allows for separation of concerns] -
 With CSS the form is separated from the page's HTML structure and content. Makes it easier to read, maintain and crawl the code.

\item [CSS has a captive audience] -
 Support for CSS development is huge. At the same time more and more libraries, tools and frameworks focus on improving and simplifing CSS development. 

\item [CSS is fast] - 
 External CSS speeds up HTML downlaod and loading compared to HTMLs with duplicated inline styles. Compared to JavaScript it also processes transitions and animations faster.

\item [CSS is fault-tolarent] - 
 Browser-unknown enhancements are simply ignored by the browser, while the remainder is still used and displayed.

\item [CSS is everywhere] - 
 Modern browsers embrace CSS and feature support by each can be easily found online.
\end{description}

% section everythingCSS (end)

\section{CSS Animation Declaration} % (fold)
\label{sec:declarationCSS}

As stated in section \ref{sec:anime_motion}, animation is about changing an element's attribute(s) over time. In CSS we can redefine it as a switch between CSS styles for a HTML element that happens gradually over time. It is stated that CSS animation should be done with {\em{}Animation} property(ies) and {\em{}Keyframe} rule(s). This may be the most efficient CSS way to accomplish animation, but CSS animation can also be achieved with {\em{}Transition} property(ies) and {\em{}selector} pattern(s)\citep{w3schoolAnime,w3schoolTrans}.


\TODO{\subsection{Animation Property and Keframe Rule}} % (fold)
\label{sub:CSS_animation_keyframe}

\begin{description}
\item [animation-name:] desiredName
\item [animation-duration:] 5s
\item [animation-timing-function:] linear | ease | ease-in | ease-out | ease-in-out | cubic-bezier(x1,y1,x2,y1) | steps(stepSize, start, end)
\item [animation-delay:] 2s
\item [animation-iteration-count:] number | infinite
\item [animation-direction:] normal | reverse | alternate| alternate-reverse
\item [animation-fill-mode:] none | forwards | backwards | both
\item [animation-play-state:] paused | running
\end{description}
HEH

\begin{description}
\item [animation:] desiredName 5s linear 2s infinite alternate none running;
\end{description}

Fine-tune the actual animation through keyframes
@keyframes rule


\begin{lstlisting}[
language=CSS,
label=list:BibACMIEEE,
caption={[Example of non-motion animation in CSS]%
Simple example of non-motion animation with animation property and keyframes rule. A working example can be found in code/nonmotionAnimationCSS.html.
}
]
div{
	animation: desiredName 4s linear 0s infinite alternate;
}

@keyframes desiredName {
	0%   {background-color:red;opacity: 1;transform: scale(1);}
	25%  {background-color:yellow;transform: scale(0.8);}
	50%  {background-color:blue;transform: scale(1);}
	75%  {background-color:green;transform: scale(1.5);}
	100% {background-color:red;opacity: 0.2;transform: scale(2);}
}
\end{lstlisting}

% subsection CSS_animation_keyframe (end)

\TODO{\subsection{Transition Property and Selector Pattern}} % (fold)
\label{sub:CSS_transition}


\begin{description}
\item [transition-property:] propertyName | all
\item [transition-duration:] 5s
\item [transition-timing-function:] linear | ease | ease-in | ease-out | ease-in-out | cubic-bezier(x1,y1,x2,y1)
\item [transition-delay:] 2s
\end{description}


\begin{lstlisting}[
language=CSS,
label=list:BibACMIEEE,
caption={[Example of non-motion transition in CSS]%
Simple example of non-motion animation with transition property and hover selector. A working example can be found in code/nonmotionTransitionCSS.html.
}
]
div{
	background-color: red;
	opacity: 1;
	transition: all 4s;
}
div:hover{
	background-color:green;
	opacity: 0.2;
	transform: scale(2);
}
\end{lstlisting}

% subsection CSS_transition (end)

% section declarationCSS (end)

\section{CSS Examples} % (fold)
\label{sec:CSS_Examples}

% section CSS_Examples (end)

\subsection{Navigation Animation} % (fold)
\label{sub:navigationCSS}

As stated in \TODO{ref section useful}, animation can help with navigation through tha page. In the examples we show two typical cases of such usage.

\TODO{\subsubsection{Hamburger Icon}} % (fold)
\label{subsub:hamburger}

\citet{vtldesign}
\citep{vtldesign}

\begin{figure}[tp]
\centering
\includegraphics[keepaspectratio,scale=0.5]{images/hamburgerVar.png}
\includegraphics[keepaspectratio,scale=0.5]{images/hamburgerMenu.png}

\caption[Hamburger Examples]{
One the left we have a screenshot of code example hamburgerVariationCSS.html, which shows some of the different ways how hamburger icons are animated. On the right side we have hamburgerMenuCSS.html, that demonstrates the showing of the menu by hovering over the hamburger icon.
\imgcredit{Screenshot taken by the authors of this survey. The code behind the pages is by extending the \cite{,} online snipets.}
}
\label{fig:storyboard}
\end{figure}



\TODO{\subsubsection{Current menu position indicator}} % (fold)
\label{subsub:menu}

% subsection navigationCSS (end)

\subsection{Loading Animation} % (fold)
\label{sub:loadingCSS}

Another use of animation is user feedback, see \TODO{ref section useful}.

\TODO{\subsubsection{Rotating Icon}} % (fold)
\label{subsub:rotation_loader}
2D and 3D

\TODO{\subsubsection{Horizontal movement}} % (fold)
\label{subsub:menu}

dots, progress bar, pulz (wave motion)

% section loadingCSS (end)

% Alexei Kruglov
%----------------------------------------------------------------
%
%  File    :  survey-SVG.tex
%
%  Author  : Alexei Kruglov, TU Graz, Austria
% 
%  Created :  01 Dec 2016
% 
%  Changed :  X Dec 2016
% 
%----------------------------------------------------------------


\section{Scalable Vector Graphics (SVG)}
\label{sect:SVG}
%\TODO{{section by A.K.}}
SVG (Scalable Vector Graphics) is an XML (Extensible Markup Language) based specification for describing two-dimensional graphics in a vector form. It is specified by W3C (World Wide Web Consortium). Initial release was made in 2001, and the latest release (1.1) was in 2011. Currently it is used in all latest major web browsers, such as Google Chrome, Mozilla Firefox and Internet Explorer, and in some vector graphics software editors, as Inkscape and Adobe Illustrator. In contrast to raster images, vector images are resolution independent, as the format itself is lossless. As mentioned before, the graphic is described in an XML textual file, which can be compressed with gzip. As a result of compression, web pages, where SVG is used load faster than those which use standard JPEG or PNG graphics solutions. Because graphics can be directly embedded into HTML, no extra HTTP requests are necessary. Above approaches lead to saving of the bandwidth, which is extremely important on the pages with higher loads. SVG has a support for different graphics effects and animations, which can be further enhanced with CSS and JavaScript. In case SVG graphic is made in a very complex node structure, browser can have problems rendering the image. In this case other technologies can be used. SVG integrates with other W3C standards such as the DOM and XSL\citep{w3schoolSVG}. 

\subsection{SVG basic graphical elements} % (fold)
\label{sub:SVG_basic_elemnts}
As mentioned before SVG graphics can be embedded directly into HTML pages
\begin{lstlisting}[
language=CSS,
label=list:BibACMIEEE,
caption={[Example of SVG embedded directyl into HTML]%
Simple example code one circle embedded directyl into HTML.
}
]
<html>
<body>

<svg width="100" height="100">
  <circle cx="50" cy="50" r="40" stroke="green" stroke-width="4" fill="yellow" />
</svg>

</body>
</html>

\end{lstlisting}
\label{list:SVGHTML}
\begin{figure}[h]
\centering
\includegraphics[keepaspectratio,scale=0.5]{images/circle.png}

\caption[SVG basic graphical elements]{
This example shows how simple to add one element (circle) embedded directyl into HTML. 
\imgcredit{Screenshot taken by the authors of this survey. The code behind the pages is by using the \citet{CircleHtml} online snippets as a base.}

}
\label{fig:circleHTML}
\end{figure}
SVG consists of different basic graphical elements, from which more complex elements can be derived. Basic elements included in SVG are:
\begin{description}
\item [<rect>] 	  -Element is used to draw a rectangle and variations of the shape of a rectangle.
\item [<circle>]      	 -Element is used to create a simple circle with color options.
\item [<line>]         	 -Element is used to create a line.
\item [<polygon>]  	 -Element is used to create a graphic that contains at least three sides.
\item [<polyline>]  	 -Element is used to provide any shape consisting of straight lines.
\item [<path>]    	 -Element is used to define a path. Participates in paths drawing strongly to use to encourage the SVG editor to create complex graphics.
\end{description}

With these simple, basic of elements, you can build a very complex shapes. With heli additional graphics functions, you can create a moving figures, but if these figures create just only SVG, the program code will be very large and difficult to read.
\begin{figure}[h]
\centering
\includegraphics[keepaspectratio,scale=0.5]{images/Car_svg.png}

\caption[Complex picture with simple SVG elements]{
Example to create one complex picture with simple SVG elements, with help geomety and color options.
\imgcredit{Screenshot taken by the authors of this survey. The code behind the pages is by using the \citet{CarSVGl} online snippets as a base.}

}
\label{fig:Circle_HTML}
\end{figure}
% SVG basic graphical elements (end)
\subsection{SVG animation } % (fold)
SVG element is a specific DOM element, which includes himself syntax standard HTML-element. SVG elements have unique tags, attributes and behaviors that enable them to determine any form that offer the opportunity to essentially produce directly to the DOM, images, and thereby benefit from the JavaScript and CSS-based manipulation. There are three main advantages to create graphics in SVG, than an image (PNG, JPEG, etc.): First, compresses incredibly well, certain terms in the SVG file format is smaller than their PNG / JPEG equivalents. Secondly, SVG graphics to scale to any resolution without loss of clarity; they look sharp on all desktop and mobile screens. Third, you can animate the individual components of the SVG graphics performance (with help JavaScript and CSS).SVG elements take some of the standard CSS properties, but not all. In addition, SVG takes a certain set of "presentation" attributes, such as fill, x and y, which also serve to determine both the SVG is visually observed. There is no functional difference between the SVG specification of style using CSS, or as an attribute - SVG specification only divide property under two. In following examples shows how SVG work with CSS together:
\begin{description}
\item [Define the imeage as vectors in SVG]
\item [Tranform the vectors with CSS or SVG]
\end{description}
\begin{lstlisting}[
language=CSS,
label=list:BibACMIEEE,
caption={[Define the imeage as vectors in SVG]%
Define the attribute for animate transorm.
}
]
<animateTransform attributeName="transform"
		attributeType="XML"
		type="translate"
		values="0 50;0 -50;"
		dur="2s"
		repeatCount="indefinite"/>

\end{lstlisting}
\label{list:animateTransorm}
\subsubsection {Create SVG stucture into HTML} % (fold)
\begin{lstlisting}[
language=CSS,
label=list:BibACMIEEE,
caption={[Create SVG stucture into HTML]%
Define svg class  with three simple {\em{}<path>} SVG elements with transform function .
}
]
<svg class="svg-icon" viewBox="0 0 20 20">
		<path id="svg-doc" d="M15.475,6.692l-4.084-4.083C11.32,
		2.538,11.223,2.5,11.125,2.5h-6c-0.413,0-0.75,0.337-0.75,
		0.75v13.5c0,0.412,0.337,0.75,0.75,0.75h9.75c0.412,
		0,0.75-0.338,0.75-0.75V6.94C15.609,6.839,15.554,6.771,
		15.475,6.692 M11.5,3.779l2.843,2.846H11.5V3.779z
		M14.875,16.75h-9.75V3.25h5.625V7c0,0.206,0.168,0.375,
		0.375,0.375h3.75V16.75z"
		transform="scale(0.25) translate(27.5)"></path>

		<path id="svg-bottom" fill="none" d="M16.471,
		5.962c-0.365-0.066-0.709,0.176-0.774,0.538l-1.843,
		10.217H6.096L4.255,6.5c-0.066-0.362-0.42-0.603-0.775-0.538
		C3.117,6.027,2.876,6.375,2.942,6.737l1.94,10.765c0.058,
		0.318,0.334,0.549,0.657,0.549h8.872c0.323,0,0.6-0.23,
		0.656-0.549l1.941-10.765C17.074,6.375,16.833,6.027,16.471,
		5.962z"
		transform="scale(0.75) translate(4,20)"></path>

		<path id="svg-deck" fill="none" d="M16.594,3.804H3.406
		c-0.369,0-0.667,0.298-0.667,0.667s0.299,0.667,0.667,
		0.667h13.188c0.369,0,0.667-0.298,0.667-0.667S16.963,
		3.804,16.594,3.804zM9.25,3.284h1.501c0.368,0,0.667-0.298,
		0.667-0.667c0-0.369-0.299-0.667-0.667-0.667H9.25c-0.369,
		0-0.667,0.298-0.667,0.667C8.583,2.985,8.882,3.284,9.25,3.284z"
		transform="scale(0.75) translate(4,20)"></path>
	</svg>

\end{lstlisting}
\label{list:SVGStuctureHTML}

\subsubsection {Define keyframe in CSS} % (fold)
\begin{lstlisting}[
language=CSS,
label=list:BibACMIEEE,
caption={[Create Keys CSS]%
Create CSS with {\em{@keyframes}} for picture visualization  .
}
]
#svg-doc{
	fill:red;
	animation: doc-delete 4s linear infinite;
}

#svg-deck{
	transform-origin: 100% 50%;
	animation: deck-rot 4s linear infinite;
}

@keyframes doc-delete {
	0% { transform: translate(75%) scale(0.25); }
	15% { transform: translate(75%) scale(0.25); }
	65% { transform: translate(85%,150%) scale(0.1) rotate(-180deg);}
	100% { transform: translate(85%,150%) scale(0.1); opacity: 0;}
}

@keyframes deck-rot {
	0% { transform: translate(-9%, 440%) rotate(0deg) scale(0.75); }
	25% { transform: translate(-9%, 440%) rotate(90deg) scale(0.75);}
	40% { transform: translate(-9%, 440%) rotate(90deg) scale(0.75);}
	65% { transform: translate(-9%, 440%) rotate(0deg) scale(0.75);}
	100% { transform: translate(-9%, 440%) rotate(0deg) scale(0.75);}

\end{lstlisting}
\label{list:Keys_CSS}

\begin{figure}[h]
\centering
\includegraphics[keepaspectratio,scale=0.5]{images/beer_svg.png}

\caption[Complete example create animation SVG with CSS]{
Example to create one complex moving animation in SVG with help of CSS.
\imgcredit{Screenshot taken by the authors of this survey. The code behind the pages is by using the \citet{beerSVG} online snippets as a base.}

}
\label{fig:SVG_CSS}
\end{figure}


\begin{figure}[h]
\centering
\includegraphics[keepaspectratio,scale=0.5]{images/icon_svg.png}

\caption[Animation from finished icons]{
Example to create animation with help prepared simple icons. 
\imgcredit{Screenshot taken by the authors of this survey. The code behind the pages is by using the \citet{trashSVG} online snippets as a base.}

}
\label{fig:SVGWithSimpleIcons}
\end{figure}

% Helmut Zöhrer
\cleardoublepage
%----------------------------------------------------------------
%
%  File    :  survey-JS.tex
%
%  Author  :  Helmut Zöhrer, TU Graz, Austria
% 
%  Created :  01 Dec 2016
% 
%  Changed :  X Dec 2016
% 
%----------------------------------------------------------------


\chapter{JavaScript (JS)}

\label{chap:JS}

\TODO{CHAPTER INTRO IS MISSING - check chapter 2 and 3}

\section{When to Use JS instead of CSS}

As a rule of thumb, JS should not be used if the same effect could be achieved with plain CSS. This is due to performance and resource management reasons. Just when CSS is stretched to its limits, JS should be used. There is a rough distinction whether to use CSS or JS for particular kinds of tasks:
\begin{itemize}
	\item Use CSS animations for simple transitions, like changing the state of a UI element.
	\item Use JS animations to get advanced effects like bouncing, stop, pause, rewind, or slow down (there is more control over animations).
	\item When choosing to animate with JS, considering using the Web Animations API or a modern framework (comfortable to work with) may be desirable.
	\item Using both CSS and JS works also well:
	\begin{itemize}
		\item perform animations with CSS
		\item control states with JS
	\end{itemize}
\end{itemize}


\section{Examples of Useful Animations with JS}

\TODO{this!}
\TODO{citing with: \citet{googleDev} or \citep{googleDev}}


\begin{lstlisting}[
language=JavaScript,
label=list:BibACMIEEE,
caption={[Some Code Snippet]%
This is a code snippet where JS is used in a meaningful way.
}
]
// create some nodes
var headline = document.createElement('h1');
var text = document.createTextNode('Dies ist eine Überschrift');
// "offline" node manipulation
headline.appendChild(text);
// adding node to DOM
document.getElementsByTagName("body")[0].appendChild(headline);
\end{lstlisting}

\cleardoublepage
%----------------------------------------------------------------
%
%  File    :  survey-concl.tex
%
%  Author  :  Keith Andrews, IICM, TU Graz, Austria
% 
%  Created :  27 May 1993
% 
%  Changed :  16 Nov 2010
% 
%----------------------------------------------------------------


\chapter{Concluding Remarks}

\label{chap:Concl}

Through the course of further investigating web animations, we realized that animations are not merely there to make a website appear more beautiful, but to carry meaning as well. So, if a user sees a hamburger icon, they should / and nowadays probably do know what this icon stands for. 

Numerous examples of coding 


\cleardoublepage
\printbibliography[heading=bibintoc]


\end{document}