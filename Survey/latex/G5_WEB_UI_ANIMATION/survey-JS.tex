%----------------------------------------------------------------
%
%  File    :  survey-JS.tex
%
%  Author  :  Helmut Zöhrer, TU Graz, Austria
% 
%  Created :  01 Dec 2016
% 
%  Changed :  X Dec 2016
% 
%----------------------------------------------------------------


\chapter{JavaScript (JS)}

\label{chap:JS}

\TODO{CHAPTER INTRO IS MISSING - check chapter 2 and 3}

\section{When to Use JS instead of CSS}

As a rule of thumb, JS should not be used if the same effect could be achieved with plain CSS. This is due to performance and resource management reasons. Just when CSS is stretched to its limits, JS should be used. There is a rough distinction whether to use CSS or JS for particular kinds of tasks:
\begin{itemize}
	\item Use CSS animations for simple transitions, like changing the state of a UI element.
	\item Use JS animations to get advanced effects like bouncing, stop, pause, rewind, or slow down (there is more control over animations).
	\item When choosing to animate with JS, considering using the Web Animations API or a modern framework (comfortable to work with) may be desirable.
	\item Using both CSS and JS works also well:
	\begin{itemize}
		\item perform animations with CSS
		\item control states with JS
	\end{itemize}
\end{itemize}


\section{Examples of Useful Animations with JS}

\TODO{this!}
\TODO{citing with: \citet{googleDev} or \citep{googleDev}}


\begin{lstlisting}[
language=JavaScript,
label=list:BibACMIEEE,
caption={[Some Code Snippet]%
This is a code snippet where JS is used in a meaningful way.
}
]
// create some nodes
var headline = document.createElement('h1');
var text = document.createTextNode('Dies ist eine Überschrift');
// "offline" node manipulation
headline.appendChild(text);
// adding node to DOM
document.getElementsByTagName("body")[0].appendChild(headline);
\end{lstlisting}