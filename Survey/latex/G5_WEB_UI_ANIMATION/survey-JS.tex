%----------------------------------------------------------------
%
%  File    :  survey-JS.tex
%
%  Author  :  Helmut Zöhrer, TU Graz, Austria
% 
%  Created :  01 Dec 2016
% 
%  Changed :  X Dec 2016
% 
%----------------------------------------------------------------


\chapter{JavaScript (JS)}

\label{chap:JS}

As an addition to plain CSS, animations can also be brought to life via JavaScript (JS) or a combination of CSS and JS. When to use which kind of implementation method is dependent on the possibilities in terms of control and effect one wants to achieve. 

\section{When to Use JS instead of CSS}

As a rule of thumb, JS should not be used if the same effect could be achieved with plain CSS. This is due to performance and resource management reasons. Just when CSS is stretched to its limits, JS should be used. There is a rough distinction whether to use CSS or JS for particular kinds of tasks:
\begin{itemize}
	\item Use CSS animations for simple transitions, like changing the state of a UI element.
	\item Use JS animations to get advanced effects like bouncing, stop, pause, rewind, or slow down (there is more control over animations).
	\item When choosing to animate with JS, considering using the Web Animations API or a modern framework (comfortable to work with) may be desirable.
\end{itemize}
Apart from this clear distinction, using both CSS and JS works also well. One could perform animations with CSS and control states with JS.
\citep{googleDev}

\section{Examples of Useful Animations with JS}

The following example taken from \citet{transformsJS} is capable of breaking the CSS-habit of isolating transformations such as scaling, rotations and transposing. To be more precise, this example allows independent animations at a time with the help of partially overlapping start and ending times as well as different easing effects. This example would not be implementable with pure CSS.

\begin{lstlisting}[
language=JavaScript,
label=transformJS,
caption={[Example of independent transformations (rotation, position)]%
This example shows independent transformations, such as scaling, rotating and changing the position of some text.
}
]
//pulsate the box using scaleX and scaleY
TweenMax.to($box, 1.2, {scaleX:0.8, scaleY:0.8, force3D:true, yoyo:true, repeat:-1, ease:Power1.easeInOut});

$("#rotation").click(function() {
rotation += 360;
TweenLite.to($box, 2, {rotation:rotation, ease:Elastic.easeOut});
});

$("#rotationX").click(function() {
rotationX += 360;
TweenLite.to($box, 2, {rotationX:rotationX, ease:Power2.easeOut});
});

$("#rotationY").click(function() {
rotationY += 360;
TweenLite.to($box, 2, {rotationY:rotationY, ease:Power1.easeInOut});
});

$("#move").click(function() {
if (wanderTween) {
wanderTween.kill();
wanderTween = null;
TweenLite.to($box, 0.5, {x:0, y:0});
} else {
wander();
}
});

//randomly choose a place on the screen and animate there, then do it again, and again.
function wander() {
var x = (($field.width() - $box.width()) / 2) * (Math.random() * 1.8 - 0.9),
y = (($field.height() - $box.height()) / 2) * (Math.random() * 1.4 - 0.7);
wanderTween = TweenLite.to($box, 2.5, {x:x, y:y, ease:Power1.easeInOut, onComplete:wander});
}
\end{lstlisting}

\label{list:transformJS}


Another example (\citet{imposJS}) which uses the combined power of CSS and JS is the following. Some random text is animated with a lively turn-around effect. It is based on the ability to split up stings into separate characters via JS.


\begin{lstlisting}[
language=JavaScript,
label=imposJS,
caption={[Example of making some text performing turn-around effect)]%
This example implements a turn-around effect of text with the help of splitting up strings into single characters and animating them individually.
}
]
tl = new TimelineLite({onUpdate:updateSlider, onComplete:onComplete, onReverseComplete:onComplete, paused:true});

//do a simple split of the text so we can animate each character (doesn't require the advanced features of SplitText, so we just use split() and join())
$text.html("<span>" + $text.html().split("").join("</span><span>").split("<span> </span>").join("<span>&nbsp;</span>") + "</span>");

//set a perspective on the container
TweenLite.set($text, {perspective:500});

//all of the animation is created in this one line:
tl.staggerTo($("#text span"), 4, {transformOrigin:"50% 50% -30px", rotationY:-360, rotationX:360, rotation:360, ease:Elastic.easeInOut}, 0.02);
}
\end{lstlisting}

\label{list:imposJS}