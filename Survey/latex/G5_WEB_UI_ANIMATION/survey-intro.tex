%----------------------------------------------------------------
%
%  File    :  survey-intro.tex
%
%  Author  :  Keith Andrews, IICM, TU Graz, Austria
% 
%  Created :  27 May 1993
% 
%  Changed :  16 Nov 2010
% 
%----------------------------------------------------------------


\chapter{Introduction}

\label{chap:Intro}

\TODO{An academic survey paper presents a survey or overview of the state of
the art in a particular field. Every chapter and every section should
have some introductory text at the beginning, like this text. Never
jump straight in to the first secion or subsection without one or more
paragraphs of introductory text.
}





\section{Not a Series of Summaries}

A survey is \emph{not} simply a series of summaries of papers.
If I have given you say 8 papers to start you off, what you should
\emph{not} do is: divide up the papers (read two each) and produce a
series of 8 unconnected paper summaries.




\section{Read All the Papers and Research Some More}

Each of you should read \emph{all} the papers and resources: both
those I gave you and those you found yourselves.
%
Make sure you search for more papers and resources yourselves. Not
just a Google search. Search the ACM \citep{ACM-DL} and IEEE
\citep{IEEE-DL} digital libraries, citeseer \citep{CiteSeer}, and
mendeley \citep{Mendeley}. You may want to use mendeley to collect
your resources or maybe maintain a .bib file within an SVN repository.

Include a list of \emph{all} the relevant papers and resources you
have found and mark those you have chosen to focus on. Make sure
\emph{all} the papers and resources you found or were given appear in
the bibliography.




\section{Dividing up the Field}

The hardest part of any survey is dividing up the field.  Look for
common concepts and threads in the papers and resources. Do they
report similar or dissimilar results? Does one paper or resource
support or contradict another?

Once you have all read all the papers: you need to construct a small
hierarchy (taxonomy) to classify the concepts appearing in the papers
and resources. Structure your survey into chapters and sections based
on your taxonomy.