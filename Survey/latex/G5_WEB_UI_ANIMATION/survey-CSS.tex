%----------------------------------------------------------------
%
%  File    :  survey-CSS.tex
%
%  Author  :  Rok Kogovšek, TU Graz, Austria
% 
%  Created :  01 Dec 2016
% 
%  Changed :  X Dec 2016
% 
%----------------------------------------------------------------


\chapter{Cascading Style Sheets (CSS)}

\label{chap:CSS}

Knowing the usefulness of animation in web UI and the correct way of animation planing are just the fundemantals for our conceptual plans. Those still need to be implemented to get the end product and here we usually hit a wall build from the various tools, that say they can all solve our problems. Even well established people in the field have stories as such to tell. Val Head actually started with animation due to an interesting Flash workshop. Flash was at that time the de facto king in its era, however as we know, that era is already dead. Nowdays we can acomplish
all we could with Flash and more with just the core parts of the web, namely HTML, CSS and JS, 

\section{Do everything you can with CSS}

As \citet{champeon2003inclusive}
 said: {\em"Leave no one behind"}.

browsers, even modern browsers, have widely varying capabilities
accessibility is for everyone, not just the disabled
it is possible to support all browsers with X/HTML and CSS


Progressive Enhancement Strategies
start with lowest common denominator (baseline)
design for semantics and structure
add features appropriate for baseline devices
warnings about display, to be hidden with CSS
“skip nav” links
add features appropriate for accessibility
alt, longdesc, title

\citet{champeon2003inclusive}



TO DO

\citet{w3school}
\citep{w3school}


\citet{vtldesign}
\citep{vtldesign}

\citet{head2016designing}
\citep{head2016designing}