%----------------------------------------------------------------
%
%  File    :  survey-CSS.tex
%
%  Author  :  Rok Kogovšek, TU Graz, Austria
% 
%  Created :  01 Dec 2016
% 
%  Changed :  X Dec 2016
% 
%----------------------------------------------------------------


\chapter{Cascading Style Sheets (CSS)}

\label{chap:CSS}

Knowing the usefulness of animation in web UI and the correct way of animation planing are just the fundemantals for our conceptual plans. Those still need to be implemented to get the end product and here we usually hit a wall build from the various tools, that say they can all solve our problems. Even well established people in the field have stories as such to tell. Val Head actually started with animation due to an interesting Flash workshop. Flash was at that time the de facto king in its era, however as we know, that era is already dead. Nowdays we can acomplish
all we could with Flash and more with just the core parts of the web, namely HTML, CSS and JS
\citep{head2016designing}.

\section{Do Everything You Can With CSS}

\label{sec:everythingCSS}

With Responsive web design (RWD) in our websites and animation being part of the design, see section \TODO{\ref{chap:Animation}}, it should be natural to use the guidelines of RWD also in animation planning. \citet{IAWEB} teaches us that one of the RWD guidelines is also Progressive enhancement, which is best described with words of the conceptual authors \citet{champeon2003inclusive}: {\em"Leave no one behind. . . .  accessibility is for everyone, not just the disabled"}. With CSS nowdays being a core part of the web and at the same time being the lowest web component that enables animation with RWD guidelines\footnote{Of course one can just use an animated image, e.q. a GIF with an image sequence, and just append it with HTML into the design. However, this image will become a static component of the design and will not follow RWD guidelines.}, one should always implement with CSS and HTML alone as much of the desired animation as possible. One has only to make sure the browser support for the animated attribute.

Other supporting arguments for use of CSS as the starting point for web UI animation beside responsiveness can be summarized with the the so called "Simple CSS Truths", a list of truths by \citet{palermoCSS} enhanced with the teachings of \citet{IAWEB}:

\begin{description}
\item [CSS allows for separation of concerns] -
 With CSS the form is separated from the page's HTML structure and content. Makes it easier to read, maintain and crawl the code.

\item [CSS has a captive audience] -
 Support for CSS development is huge. At the same time more and more libraries, tools and frameworks focus on improving and simplifing CSS development. 

\item [CSS is fast] - 
 External CSS speeds up HTML downlaod and loading compared to HTMLs with duplicated inline styles. Compared to JavaScript it also processes transitions and animations faster.

\item [CSS is fault-tolarent] - 
 Browser-unknown enhancements are simply ignored by the browser, while the remainder is still used and displayed.

\item [CSS is everywhere] - 
 Modern browsers embrace CSS and feature support by each can be easily found online.
\end{description}

% section everythingCSS (end)

\section{CSS Animation Declaration} % (fold)
\label{sec:declarationCSS}

As stated in section \TODO{\ref{chap:Animation}}, animation is about changing an element's attribute(s) over time. In CSS we can redefine it as a switch between CSS styles for a HTML element that happens gradually over time. It is stated that CSS animation should be done with {\em{}Animation} property(/ies) and {\em{}Keyframe} rule(s). However, that is not the only CSS way to accomplish animation, but of course the most efficient way \citep{w3school}.


\TODO{\subsection{Animation Property and Keframe Rule}} % (fold)
\label{sub:CSS_animation_keyframe}

% subsection CSS_animation_keyframe (end)

\TODO{\subsection{Transition Property}} % (fold)
\label{sub:CSS_transition}

% subsection CSS_transition (end)

\TODO{\subsection{Selector Pattern}} % (fold)
\label{sub:CSS_transition}

% subsection CSS_transition (end)



% section declarationCSS (end)

\section{CSS Examples} % (fold)
\label{sec:CSS_Examples}

% section CSS_Examples (end)

\subsection{Navigation Animation} % (fold)
\label{sub:navigationCSS}

As stated in \TODO{ref section useful}, animation can help with navigation through tha page. In the examples we show two typical cases of such usage.

\TODO{\subsubsection{Hamburger Icon}} % (fold)
\label{subsub:hamburger}

\citet{vtldesign}
\citep{vtldesign}

\TODO{\subsubsection{Current menu position indicator}} % (fold)
\label{subsub:menu}

% subsection navigationCSS (end)

\subsection{Loading Animation} % (fold)
\label{sub:loadingCSS}

Another use of animation is user feedback, see \TODO{ref section useful}.

\TODO{\subsubsection{Rotating Icon}} % (fold)
\label{subsub:rotation_loader}
2D and 3D

\TODO{\subsubsection{Horizontal movement}} % (fold)
\label{subsub:menu}

dots, progress bar, pulz (wave motion)

% section loadingCSS (end)