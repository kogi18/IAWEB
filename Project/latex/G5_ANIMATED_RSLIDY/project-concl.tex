%----------------------------------------------------------------
%
%  File    :  survey-concl.tex
%
%  Author  :  Keith Andrews, IICM, TU Graz, Austria
% 
%  Created :  27 May 1993
% 
%  Changed :  16 Nov 2010
% 
%----------------------------------------------------------------


\chapter{Concluding Remarks}

\label{chap:Concl}

In the end the pure \textit{rSlidy} solution amounted over 800 new CSS lines of 
code (LOC) and over 500 new
JavaScript LOC. There is still some room for improvement that can be done. Nice 
examples are ideas we disscussed at the planning stage, such as adding a 
canvas, to have marker support or animated CSS text styles for getting 
attention on particular parts of text. Another missing part that the original 
had, was a TypeScript implementation for working with Grunt.

While responsive web design was used and tested for, most problems were 
successfully removed. However testing on actual devices devices besides using 
developer tools and online testing sites, revealed some small design errors. 
Our group tested it out on Samsung Galaxy S2 and S6, Iphone 5, 6 and 6 plus, 
Ipad and Ipad Pro. The testing was also done on different desktop screen 
resolutions. The found problems should be corrected, however it leaves room of 
uncertanties that due to time constraints were not tested for. This just proves 
the responsive web design should not be underestimated, specifically when an 
existing project one has no preknowledge is being developed.


 Despite the room for improvement, our opinion is united, that we achieved 
better user experience for slide presentations and implemented interesting 
animation solutions. On that note we also found interesting issues with CSS 
animation, such as the keyframe for same element issue, preloading of files and 
HTML rendering order or the deep dependence of CSS animation on the HTML 
rendering tree. Achieving our goal and learning of new issues, we rate this 
project as a success.