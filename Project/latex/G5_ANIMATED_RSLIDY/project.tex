%----------------------------------------------------------------
%
%  File    :  project.tex
%
%  Author  :  Keith Andrews, IICM, TU Graz, Austria
%
%  Created :  24 Mar 2010
%
%  Changed :  06 Dec 2016
%
%----------------------------------------------------------------


\documentclass[11pt,onecolumn,twoside]{report}

\usepackage[          % set page and margin sizes
  a4paper,
  twoside,
  top=5mm,
  bottom=10mm,
  inner=15mm,
  outer=15mm,
  bindingoffset=10mm,
  head=10mm,
  foot=10mm,
  headsep=15mm,
  footskip=15mm,
  includeheadfoot,
]{geometry}
% A4 is 210 x 297 mm



\usepackage{txfonts}            % new times fonts
\usepackage{relsize}            % relative font sizes \smaller \larger
\usepackage{float}              % H for float placement
\usepackage{setspace}           % line spacing

\usepackage[T1]{fontenc}        % 8-bit output chars (must be before inputenx)
\usepackage[utf8]{inputenx}     % input char encoding

\usepackage{textcomp}           % symbols such as \texttimes and \texteuro
\usepackage{latexsym}

\usepackage{xspace}
\usepackage{etoolbox}           % for \newrobustcmd
\usepackage{makecmds}           % for \makecommand


\usepackage[english,austrian,british]{babel}


\usepackage[bf,sf]{titlesec}



\setlength{\textfloatsep}{10mm plus 2mm minus 1mm}
\setlength{\floatsep}{10mm plus 2mm minus 1mm}
\setlength{\intextsep}{10mm plus 2mm minus 1mm}

\setlength{\dbltextfloatsep}{10mm plus 2mm minus 1mm}
\setlength{\dblfloatsep}{10mm plus 2mm minus 1mm}

\setlength{\abovecaptionskip}{4mm plus 2mm minus 1mm}
\setlength{\belowcaptionskip}{0mm}




% use caption and subfig (caption2 and subfigure are now obsolete)

\usepackage[
  position=bottom,
  margin=1cm,
  font=small,
  labelfont={bf,sf},
  format=hang,
  indention=0mm,
]{caption}
% ]{caption,subfig}

\usepackage{subcaption}

\captionsetup[subfigure]{
  margin=0pt,
  parskip=0pt,
  hangindent=0pt,
  indention=0pt,
  singlelinecheck=true,
}




% fancyhdr to make nice headers and footers
% and deal with long chapter names

\usepackage{fancyhdr}         % headers and footers
\pagestyle{fancy}             % must call to set defaults before redefining

\renewcommand{\chaptermark}[1]{%
  \markboth{\thechapter.\ #1}{}
}
\renewcommand{\sectionmark}[1]{%
  \markright{\thesection.\ #1}
}
\renewcommand{\headrulewidth}{0mm}
\renewcommand{\footrulewidth}{0mm}
\newcommand{\headlook}{\sffamily}
\fancyhf{}
\fancyhead[LE,RO]{\thepage}
\fancyhead[LO]{%
\parbox[t]{0.8\textwidth}{\headlook\nouppercase{\rightmark}}
}
\fancyhead[RE]{%
\parbox[t]{0.8\textwidth}{\raggedleft\headlook\nouppercase{\leftmark}}
}


%\fancypagestyle{plain}{%   redefine plain style, but doesn't work
%  \fancyhf{}    % clear all header and footer fields
%  \fancyfoot[C]{\headlook \thepage} % except the center
%  \renewcommand{\headrulewidth}{0pt}
%  \renewcommand{\footrulewidth}{0pt}
%}



\usepackage{xcolor}
\definecolor{darkgreen}{rgb}{0.0,0.2,0.0}
\definecolor{darkblue}{rgb}{0.0,0.0,0.2}
\definecolor{darkred}{rgb}{0.2,0.0,0.0}
\definecolor{verylightgrey}{gray}{0.95}
\definecolor{lightgrey}{gray}{0.9}
\definecolor{black}{gray}{0.0}


\usepackage{tabularx}


\usepackage{listings}                 % for listings of source code
\usepackage{calc}                     % for calulation below

\makeatletter
\newlength{\numwidth}%
\setlength{\numwidth}{\widthof{\normalfont{\lst@numberstyle{99}}}}% Up to 2-digit (99) line numbers
\def\lst@PlaceNumber{%
  \makebox[\numwidth+1em][l]{%
    \makebox[\numwidth][r]{\normalfont\lst@numberstyle{\thelstnumber}}%
  }%
}
\makeatother

\lstset{                              % set parameters for listings
  floatplacement=tp,                  % default float placement
  numberbychapter,
  inputencoding=utf8,
  language=,                          % empty = plain text
  basicstyle=\small\ttfamily,
  tabsize=2,
  xleftmargin=1cm,
  xrightmargin=1cm,
  frame=none,
  framexleftmargin=0mm,
  rulesepcolor=\color{verylightgrey},
  numbers=none,
  numberstyle=\scriptsize,
  numbersep=2ex,
  breaklines,
  showtabs=false,
  showspaces=false,
  showstringspaces=false,
  keywordstyle=\bfseries,
  identifierstyle=,
  stringstyle=,
  captionpos=b,
  abovecaptionskip=\abovecaptionskip,
  belowcaptionskip=\belowcaptionskip,
  aboveskip=\floatsep,
  belowskip=\floatsep,
  extendedchars=true,
  literate=%
    {©}{{\textcopyright}}1
    {€}{{\texteuro}}1
    {Ö}{{\"O}}1
    {Ä}{{\"A}}1
    {Ü}{{\"U}}1
    {ß}{{\ss}}1
    {ö}{{\"o}}1
    {ä}{{\"a}}1
    {ü}{{\"u}}1,       % map some utf8 chars to replacements
}


\lstdefinelanguage{biblatex}   % based on biblatex v 2.7a from 2013-07-14
{
  keywords={%
    @article,@book,@mvbook,@inbook,@bookinbook,@suppbook,%
    @booklet,@collection,@mvcollection,@incollection,@suppcollection,%
    @manual,@misc,@online,@patent,@periodical,@suppperiodical,%
    @proceedings,@mvproceedings,@inproceedings,@reference,@mvreference,%
    @inreference,@report,@set,@thesis,@unpublished,@xdata,%
    @conference,@electronic,@mastersthesis,@phdthesis,@techreport,@www,%
    @artwork,@audio,@bibnote,@commentary,@image,@jurisdiction,@legislation,%
    @legal,@letter,@movie,@music,@performance,@review,@software,%
    @standard,@video%
  },
  comment=[l][\itshape]{@comment},
  sensitive=false,
}


\usepackage[short]{datetime}   % load datetime *after* babel, requires fmtcount
% \newdateformat{britdate}{%
% \ordinaldate{\THEDAY} \,\monthname[\THEMONTH] \THEYEAR
% }
\newdateformat{keithdate}{%
\twodigit{\THEDAY}~\shortmonthname[\THEMONTH]~\THEYEAR
}


\usepackage[hyphens,obeyspaces]{url}
\def\UrlFont{\smaller\ttfamily}



\usepackage[
  autostyle,
  english=british,
  threshold=0,
  thresholdtype=lines,
]{csquotes}
\renewcommand{\mkcitation}[1]{\space#1}

\newenvironment*{smallquote}          % smaller text within a block quote
  {\quote\smaller}
  {\endquote}
\SetBlockEnvironment{smallquote}

% put quotation marks around block quotes
% \renewenvironment{quoteblock}{\openautoquote}{\closeautoquote}

% I prefer double quotes as outer
\DeclareQuoteStyle[keithbritish]{british}%  [variant]{style}
  {\textquotedblleft}%                      opening outer mark
  {\textquotedblright}%                     closing outer mark
  [0.05em]%
  {\textquoteleft}%                         opening inner mark
  {\textquoteright}%                        closing inner mark

\setquotestyle[keithbritish]{british}



\usepackage[
  backend=biber,
  bibstyle=authoryear-ka,
  citestyle=authoryear-ka,
  sorting=nyt,
  useprefix,                   % van and von are part of second name
  mergedate=false,             % only for authoryear style
  dashed=false,                % only for authoryear style
  abbreviate=false,
  maxcitenames=2,              % if more than two authors, then use et al
  mincitenames=1,              % if exceeds 2 authors, then use 2
  maxbibnames=99,              % print all authors in biblio
  uniquename=init,
  hyperref=true,
  backref=true,
  backrefstyle=two,
  natbib=true,
  sortlocale=en,
]{biblatex}



% set for csquotes, but \autocite only available
% after biblatex is loaded
\SetCiteCommand{\autocite}    % or maybe \parencite

% more space between entries in bib
\setlength\bibitemsep{1.5\itemsep}


% remove URL: from in front of URLs
\DeclareFieldFormat{url}{\url{#1}}
\DeclareFieldFormat{doi}{\doi{#1}}
\DeclareFieldFormat{isbn}{\isbn{#1}}
\DeclareFieldFormat{issn}{\issn{#1}}

% suppress urldate field
\DeclareSourcemap{
  \maps[datatype=bibtex]{
    \map{
      \step[fieldset=urldate, null]
    }
  }
}

% for article titles
\DeclareFieldFormat{title:article}{\emph{#1}\midsentence}

\DefineBibliographyStrings{british}{%
  january          = {Jan},
  february         = {Feb},
  march            = {Mar},
  april            = {Apr},
  may              = {May},
  june             = {Jun},
  july             = {Jul},
  august           = {Aug},
  september        = {Sep},
  october          = {Oct},
  november         = {Nov},
  december         = {Dec},
}



% \bibliography{kandrews,latex,writing,inm-plag}

%\addbibresource{writing.bib}
%\addbibresource{latex.bib}
%\addbibresource{kandrews.bib}
%\addbibresource{ivis.bib}
\addbibresource{G5.bib}
\addbibresource{images/imgLocations.bib}
\addbibresource{listings/codeSource.bib}




\usepackage{ifpdf}

\ifpdf
  % pdflatex
  \usepackage[pdftex]{graphicx}
  \DeclareGraphicsExtensions{.pdf,.jpg,.png}
  \pdfcompresslevel=9
  \pdfpageheight=297mm
  \pdfpagewidth=210mm
  \usepackage[         % hyperref should be last package loaded
    unicode,
    pdftex,
    pdftitle={Animated rslidy: Responsive HTML5 Slide Decks},
    pdfsubject={},
    pdfauthor={Rok Kogovšek, Alexei Kruglov, Fernando Pulido Ruiz, and Helmut Zöhrer},
    pdfkeywords={project, IAWEB, animation, UI, web},
    bookmarks,
    bookmarksnumbered,
    linktocpage,
    colorlinks,
    linkcolor=darkred,
    anchorcolor=red,
    citecolor=darkgreen,
    urlcolor=darkblue,
    pdfview={FitH},
    pdfstartview={Fit},
    pdfpagemode=UseOutlines,       % open bookmarks in Acrobat
    plainpages=false,              % avoids duplicate page number problem
    pdfpagelabels,                 % avoids duplicate page number problem
    breaklinks=true,               % allow links exceeding a single line
  ]{hyperref}

\else
  % latex
  % should never have to run latex, since l2h now understands pdflatex .aux
  \usepackage[dvips]{graphicx}
  \usepackage[dvips]{hyperref}
  \DeclareGraphicsExtensions{.eps}
\fi





% \liintro list item intro is a style used when list items have an
% introduction phrase (say in italics) followed by a colon.
\newcommand{\liintro}[1]{\emph{#1}}


\newcommand{\imgcredit}[1]
{\smaller{}[#1]}



\newcommand{\copyrightACM}
{%
Copyright \copyright\ by the Association for Computing Machinery, Inc.%
}




\newcommand{\daymonthyear}[3]
{%
\twodigit{#1}\hspace{0.7ex}\nolinebreak[2]\shortmonthname[#2]\hspace{0.7ex}\nolinebreak[2]#3%
}


\newcommand{\monthyear}[2]
{%
\shortmonthname[#1]\hspace{0.7ex}\nolinebreak[2]#2%
}


\newcommand{\yearmonthday}[3]
{%
\twodigit{#3}\hspace{0.7ex}\nolinebreak[2]\shortmonthname[#2]\hspace{0.7ex}\nolinebreak[2]#1%
}


\newcommand{\yearmonth}[2]
{%
\shortmonthname[#2]\hspace{0.7ex}\nolinebreak[2]#1%
}



% link to Amazon or
% http://worldcatlibraries.org/wcpa/isbn/[ISBN number]

\newrobustcmd{\isbn}[1]
{%
{%
\ifpdf
{\smaller ISBN}
\href{http://www.amazon.com/exec/obidos/ASIN/#1/keithandrewshcic}{#1}%
\else
{\smaller ISBN}
#1%
\fi
}%
}



% ISSN
% http://www.bl.uk/services/bibliographic/issn.html
% 8 digits, should be printed xxxx-xxxx
% e.g. 0020-0190 is Information Processing Letters, Elsevier
%
% Lookup services:
% http://kmittlib.lib.kmutt.ac.th:81/search/i?SEARCH=0020-0190
% http://worldcatlibraries.org/wcpa/issn/0020-0190

\newrobustcmd{\issn}[1]
{%
{%
\ifpdf
{\smaller ISSN}
\href{http://worldcatlibraries.org/wcpa/issn/#1}{#1}%
\else
{\smaller ISSN}
#1%
\fi
}%
}



% DOIs  http://www.doi.org/  e.g.
% doi:10.1038/nature723
% http://dx.doi.org/10.1038/nature723

\newrobustcmd{\doi}[1]
{%
{%
\def\UrlFont{\rmfamily}
\ifpdf                                   % pdflatex
\href{http://dx.doi.org/#1}{doi:\protect\nolinkurl{#1}}%
\else                                    % latex
doi:\protect\nolinkurl{#1}%
\fi
}%
}





\newrobustcmd{\website}[1]
{%
\ifpdf                                  % pdflatex
\href{http://#1/}{\nolinkurl{#1}}%
\else                                   % latex
\nolinkurl{#1}%
\fi
}




\newcommand{\news}[1]
{%
\ifpdf
\href{news:#1}{\nolinkurl{#1}}
\else
\nolinkurl{#1}%
\fi
}








% based on url package
% define styles for class, file, and variable names
% which break nicely at line breaks

% make the macros robust so they work inside captions, etc

\newcommand{\ttname}{\begingroup \smaller\urlstyle{tt}\Url}
\newcommand{\rmname}{\begingroup \smaller\urlstyle{rm}\Url}
\newcommand{\sfname}{\begingroup \smaller\urlstyle{sf}\Url}


% cname is for class names
\newrobustcmd{\cname}[1]{\sfname{#1}}

% fname is for file names and directory names
\newrobustcmd{\fname}[1]{\ttname{#1}}

% vname is for variable names, domain names, email addresses
\newrobustcmd{\vname}[1]{\ttname{#1}}



% Euro symbol
\newcommand{\euro}{\texteuro\,}

% times symbol
\newcommand{\timessym}{\texttimes\,}

% approx symbol
\newcommand{\approxsym}{\ensuremath\approx\,}

% plusminus symbol
\newcommand{\plusminussym}{\textpm\,}

% not equal symbol
\newcommand{\neqsym}{\ensuremath{\neq\,}}

% rightarrow symbol
\newcommand{\rightarrowsym}{\ensuremath\rightarrow\,\,}




\newcommand{\TODO}[1]
{
{\textcolor{red}{[TODO: #1]}}
}



\newcommand{\fullh}{18cm}         % height of figures for 1 per page
\newcommand{\halfh}{9.5cm}        % height of figures for 2 per page
\newcommand{\thirdh}{6cm}         % height of figures for 3 per page


\tolerance=400 
  % makes some lines with lots of white space, but      
  % tends to prevent words from sticking out in the margin





\definecolor{lightgray}{rgb}{.9,.9,.9}
\definecolor{darkgray}{rgb}{.4,.4,.4}
\definecolor{purple}{rgb}{0.65, 0.12, 0.82}
\lstdefinelanguage{JavaScript}{
	keywords={break, case, catch, continue, debugger, default, delete, do, else, false, finally, for, function, if, in, instanceof, new, null, return, switch, this, throw, true, try, typeof, var, void, while, with},
	morecomment=[l]{//},
	morecomment=[s]{/*}{*/},
	morestring=[b]',
	morestring=[b]",
	ndkeywords={class, export, boolean, throw, implements, import, this},
	keywordstyle=\color{blue}\bfseries,
	ndkeywordstyle=\color{darkgray}\bfseries,
	identifierstyle=\color{black},
	commentstyle=\color{purple}\ttfamily,
	stringstyle=\color{red}\ttfamily,
	sensitive=true
}






\lstset{
	backgroundcolor=\color{lightgray},
	float=tp,
	xleftmargin=1cm,
	xrightmargin=1cm,
    framexleftmargin=1mm,
	extendedchars=true,
	basicstyle=\footnotesize\ttfamily,
	showstringspaces=false,
	showspaces=false,
	numbers=left,
	numberstyle=\footnotesize,
	numbersep=9pt,
	tabsize=2,
	breaklines=true,
	showtabs=false,
	captionpos=b
}



\begin{document}

\keithdate

\normalsize
\pagestyle{empty}         % for preliminary pages (no numbers shown)
\pagenumbering{Roman}     % for pdf labels




\begin{titlepage}

\begin{center}
{\Large \sffamily \bfseries Animated rSlidy \\ Responsive HTML5 Slide Decks}

\vspace{1cm}

% {\large\sffamily Keith Andrews}

{\large\sffamily Group 5}

\vspace{5mm}

{\large\sffamily Rok Kogovšek, Alexei Kruglov, Fernando Pulido Ruiz, and Helmut Zöhrer}

\vspace{1cm}

% Institute for Information Systems and Computer Media (IICM), \\
% Graz University of Technology \\
% A-8010 Graz, Austria \\[1cm]


{\large
706.041 Information Architecture and Web Usability WS 2016 \\
Graz University of Technology \\
A-8010 Graz, Austria  \\[1cm]
}

\vspace{1cm}

% {\large 22 Nov 2016}

{\large \today}


\end{center}



\vspace{2cm}

\begin{quote}
\begin{center}
{\large\sffamily\bfseries Abstract}
\end{center}

This project report tries to give insights into the implementation of refinements of the already existent presentation software \textit{rslidy} provided by Keith Andrews of TUG.. It includes direct comparisons of the initial and the new version(s) in terms of design and functionality. Not only this contrast, but also the particular ways of implementing certain new features are listed and discussed.The focus of the project was to make the already working version of the web slideshow application closer to well spread desktop solutions. This entitled improving the user interface to be more interactive, responsive and user friendly. A big focus of the project was on animation solutions.


\end{quote}

\vfill

\begin{center}
{\small\sffamily \copyright ~ Copyright 2017 by the author(s),
except as otherwise noted.}

\vspace{2mm}
{\footnotesize\sffamily This work is placed under a
Creative Commons Attribution 4.0 International
(\href{https://creativecommons.org/licenses/by/4.0/}{CC BY 4.0}) licence. It uses the LaTex template from "Writing a Survey Paper" by Keith Andrews, used under CC BY 4.0 / Desaturated from original
}
\end{center}

\end{titlepage}




\cleardoublepage
\pagestyle{plain}
\pagenumbering{roman}



{
\setlength{\parskip}{3pt plus 3pt minus 3pt}     % compact tables of contents
\tableofcontents
\addcontentsline{toc}{chapter}{Contents}

\cleardoublepage
\listoffigures
\addcontentsline{toc}{chapter}{List of Figures}

%\cleardoublepage
%\listoftables
%\addcontentsline{toc}{chapter}{List of Tables}

\cleardoublepage
\renewcommand{\lstlistlistingname}{List of Listings}
\lstlistoflistings
\addcontentsline{toc}{chapter}{List of Listings}
}

% define CSS source code coloring
\lstdefinelanguage{CSS} 
{
alsoletter={<>/.\#},
keywords={},
% CSS properties
keywords={ border,transform,origin,transition,duration,timing-function,animation,color,background,margin,padding,font,weight,display,position,top,left,right,bottom,list,style,size,white,space,min,width,opacity,height,delay, content,radius, stroke, fill, cx, cy,attributeName, attributeType, values, type, dur, repeatCount, viewBox, d},
% CSS rules
keywords=[2]{@keyframes},
% CSS selectors
keywords=[3]{hover,before,after, nth, child},
% HTML attributes
keywords=[4]{class, alt, id, role, href,
% calss names for linking it together with CSS
desiredName, spin, rotating, bar, loading, dots, move, pulz, delete, rot, docs, decks },
% HTML tags
keywords=[5]{
  >, />, <html>, , </html> ,
  % body
  <body>, </body>,
  % Divs
  div, </div, <div, </div>, <div></div>,
  % Paragraphs
  </p, <p, </p>,
  % scripts
  </script, <script,
  % navigation
  a, <a, </a>,
  nav, <nav>, </nav>,
  ul, <ul>, </ul>,
  li, <li>, </li>, <li><a, </a></li>,
  % More tags...
  <canvas, /canvas>, <svg, <rect, <animateTransform, </rect>, </svg>, <video, <source, <iframe, </iframe>, </video>, <image, </image>, <header, </header, <article, </article, <circle, <path, </path>
  },
  keywords=[6]{
    .hamburger, .paperNav, container, .inner,
    .topBotomBordersOut, .borderYtoX, .highlightTextOut, .circleLoader, .squareloader,
    .dotsLoader, .progressbarloader,.pulzLoader, \#svg, doc, deck, .container
  },
sensitive=false,
mathescape=true,
morecomment=[l]{//}, 
morecomment=[s]{/*}{*/},
morestring=[b][\color{darkred}]",
morestring=[b][\color{darkred}]',
% morecomment=[s][\color{darkred}]{\ .}{\ },
keywordstyle=\color{blue},
keywordstyle=[2]\color{brown},
keywordstyle=[3]\color{orange},
keywordstyle=[4]\color{violet},
keywordstyle=[5]\color{teal},
keywordstyle=[6]\color{darkred}
}

\cleardoublepage
\pagestyle{headings}        % for main pages
\pagenumbering{arabic}

\cleardoublepage
%----------------------------------------------------------------
%
%  File    :  survey-intro.tex
%
%  Author  :  Keith Andrews, IICM, TU Graz, Austria
% 
%  Created :  27 May 1993
% 
%  Changed :  06 Dec 2016
% 
%----------------------------------------------------------------


\chapter{Introduction}

\label{chap:Intro}

Our group was assigned with the task of refining the already existent presentation software \textit{rSlidy}. Therefore we first made some usability tests on our own in order to become familiar with the software and to find room for improvement. We interactively agreed with our instructor on features that needed implementation and those which would be nice to have, but not necessary.

The final submission comprises two versions of the new \textit{rSlidy}. One which is completely independent and one which uses two third party libraries. The independent one might not look as impressive in some scenarios, but is free of external code. The other version is arguably better design-wise, but relies on third party libraries, which was not in favor of the instructor. 

%HEre we tell we prepared a rSlidy only upgrade and a rSlidy supported by 3rd party software

% Fernando Pulido Ruiz
\cleardoublepage
%----------------------------------------------------------------
%
%  File    :  survey-intro.tex
%
%  Author  :  Keith Andrews, IICM, TU Graz, Austria
% 
%  Created :  27 May 1993
% 
%  Changed :  03 Feb 2017
% 
%----------------------------------------------------------------


\chapter{rSlidy}

\label{chap:rslidy}

\textit{rSlidy} is a pure HTML, CSS and JavaScript solution for slide 
presentations in ongoing development at TUG. The project seems to have been 
started by Markus Schofnegger according to the author lines in the core 
JavaScript file. It already had many upgrades in form of projects before it 
came to us for the next upgrade. The average user does not need much more then 
basic HTML coding knowledge to use it, since every presentation is composed in 
a single HTML file with references to the supporting JavaScript core and CSS 
files. The slides themselves are defined as divs with slide class, with the 
content being encapsulate in standard HTML tags, chosen by the user for uniform 
design over the slides. Custom CSS and JavaScript code can be added after the 
neaded resource files.

\begin{minipage}{\linewidth}
\begin{lstlisting}[
language=CSS,
label=list:rslidySetup,
caption={[Connecting rSlidy Resources]%
	By adding the following lines to the head of the HTML slide file, the 
slideshow is initialize on load event. 
}
]
<link rel="stylesheet" href="css/reset.css"/>
<link rel="stylesheet" href="css/normalise.css"/>
<link rel="stylesheet" href="css/rslidy.css" />
<link rel="stylesheet" href="css/slides-default.css"/>

<script src="js/rslidy.js"> </script>
\end{lstlisting}
\end{minipage}

After observing the faults of the project, the goal became to create more user 
friendly plug \& play presentation functions, mainly by animation integration. 
The upgraded rslidy version setup is the same by loading the needed JS and CSS 
in the presentation file, however an addition rslidy-animation.css is needed, 
since it defines the newly added animations, that had no previous component. 
While developing solutions we also found some nice 3rd party components, that 
are presented in enhanced version of our solution. This is also the reason for 
providing 2 solution versions, one pure \textit{rSlidy} solution and an 
enhanced solution.

% section  (end)

% Helmut Zöhrer
\cleardoublepage
%----------------------------------------------------------------
%
%  File    :  survey-design.tex
%
%  Author  :  Keith Andrews, IICM, TU Graz, Austria
% 
%  Created :  27 May 1993
% 
%  Changed :  03 Feb 2017
% 
%----------------------------------------------------------------


\chapter{Changes in design}

\label{chap:design}

helmuts status bar and so on . hover effect should go under animation probably

% Alexei Kruglov
%----------------------------------------------------------------
%
%  File    :  survey-images.tex
%
%  Author  :  Keith Andrews, IICM, TU Graz, Austria
% 
%  Created :  27 May 1993
% 
%  Changed :  03 Feb 2017
% 
%----------------------------------------------------------------


\chapter{Image magnification}

\label{chap:images}

pop up and tab version

% Rok Kogovšek
\cleardoublepage
%----------------------------------------------------------------
%
%  File    :  survey-animation.tex
%
%  Author  :  Keith Andrews, IICM, TU Graz, Austria
% 
%  Created :  27 May 1993
% 
%  Changed :  03 Feb 2017
% 
%----------------------------------------------------------------


\chapter{Animated Slideshow}

\label{chap:animated}

The biggest downside of the original \textit{rSlidy} when compared to alternative solutions, with focus on well spread desktop solutions, would be its static state. Namely the presentation flow achieved with the application was an immediate switch between states or slides. Even the user interface was behaving similarly in a state-switching way. In the design changes the effects of hiding elements with hover and transition CSS elements also include a better user experience. This follows the observations from the web animation survey \citet{WebAnime}, were the importance of animation for user experience was stressed out. With knowledge gathered from the mentioned survey we enhanced both the user interface as well the presentation flow by incorporating animation with CSS and JavaScript. For better overview, the animated old elements, that were already present in the original \textit{rSlidy}, have CSS added near the original class definitions, while the new concepts, sections \ref{sec:initialization} and \ref{sec:slide_transitions}, are defined in the new CSS file \textit{rslidy-animation.css}.

\section{Initialization Progress Animation} % (fold)
\label{sec:initialization}

\begin{figure}[tp]
	\centering
	\includegraphics[width = .2\textwidth]{images/pulzLoad.png}
	\includegraphics[width = .4\textwidth]{images/loading.png}	
	\caption[Loader]{
		Screenshot of the loading animations midway. On the left we see the inspirational pulz loader, while on the right we see the resulting rearrangement for \textit{rSlidy}.
		\imgcredit{Screenshot taken by the authors of this report.}
	}
	\label{fig:loading}
\end{figure}

% section initialization (end)

\section{Button Animation} % (fold)
\label{sec:button_animation}

Similar to an animated hamburger icon, which changes its shape on click, the buttons within the status bar of \textit{rSlidy} are animated now as well. Listing \ref{list:buttonflip} shows how the animated flip was created. These style changes and the button's text changed to an "X" lead to a simple and intuitive animation of a button which turns around to change its functionality. While the animation is done in CSS, the text change is done by the JavaScript setTimeout, to change the value halfway through the animation, when the text is not visible.
\begin{minipage}{\linewidth}
	\begin{lstlisting}[
	language=CSS,
	label=list:buttonflip,
	caption={[Flip Button Animation] Implementation of the animation of the buttons in the status bar %
	\imgcredit{The 
	code example is based 
	on the users' implementation.}
	}
	]
#button-overview, #button-toc, #button-menu{ 
	animation-duration: 0.3s; 
	animation-timing-function: ease-in-out;
	animation-fill-mode: forwards;
	animation-name: flip2Face;
}

#button-overview.clicked, #button-toc.clicked, #button-menu.clicked{
	animation-name: flip2Back; 
	transform: rotateY(180);
}
	\end{lstlisting}
\end{minipage}

\subsection{Same Element Animation Issue} % (fold)
\label{sub:same_element_animation_issue}

% subsection same_element_animation_issue (end)

% section button_animation (end)

\section{Hiding Elements} % (fold)
\label{sec:hiding_elements}

While hiding elements on hover elements by itself already raises the user experience, just by adding simple transition or animation CSS element we can direct the switch between states into a smooth way to enhance the user workflow. For better control we also used the Cubic Bezier function, with the values shown in figure \ref{fig:cubic-bezier}.

\begin{figure}[tp]
	\centering
	\includegraphics[width = .4\textwidth]{images/cubic-bezier.png}
	
	\caption[Cubic Bezier Function]{
		The values of the Cubic Bezier function used for smoother hiding of elements and its plot representation
		\imgcredit{Screenshot of the representation in Chrome Developer Tools taken by the authors of this report.}
	}
	\label{fig:cubic-bezier}
\end{figure}

% section hiding_elements (end)

\section{Preview Scrolling} % (fold)
\label{sec:preview_scrolling}

% section preview_scrolling (end)

\section{Slide Transitions} % (fold)
\label{sec:slide_transitions}

\subsection{Solution Limitations} % (fold)
\label{sub:solution_limitations}

% subsection solution_limitations (end)

\begin{figure}[tp]
	\centering
	\includegraphics[width = .9\textwidth]{images/transitions.png}
	
	\caption[Slide Transition Diagram]{
		Diagram of \textit{rSlidy} included transitions and their flow. Red slides are the previous slides, while green are active slides. The position of active and previous slide is shown as relative position to the other.
		\imgcredit{Diagram is prepared by the authors of this report.}
	}
	\label{fig:transitions}
\end{figure}

% section slide_transitions (end)

\cleardoublepage
%----------------------------------------------------------------
%
%  File    :  survey-concl.tex
%
%  Author  :  Keith Andrews, IICM, TU Graz, Austria
% 
%  Created :  27 May 1993
% 
%  Changed :  16 Nov 2010
% 
%----------------------------------------------------------------


\chapter{Concluding Remarks}

\label{chap:Concl}

CSS LOC 800+
JS LOC 500+



Still room for improvement - Some tasks that were disscussed at the planning stage but deemd not needed in this stage

TypeScript for Grunt
Including Subheadings with in TOC on heading hover
Animated CSS text tyles
Canvas to actually draw on the slide - Marker support


Achieved better user experience for slide shows

Interesting problems
Same animation on same element cannot be triggered in subclass - either need additional element or reversed keframe

Element rendering knowledge is needed for animation planning - the display:none problem.

RWD testing should not be underestimated --- HERE Fernandos text

Javascript preloading and initialization tricky with HTML rendering

\cleardoublepage
\printbibliography[heading=bibintoc]


\end{document}