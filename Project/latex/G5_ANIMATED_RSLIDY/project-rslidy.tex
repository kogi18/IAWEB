%----------------------------------------------------------------
%
%  File    :  survey-intro.tex
%
%  Author  :  Keith Andrews, IICM, TU Graz, Austria
% 
%  Created :  27 May 1993
% 
%  Changed :  03 Feb 2017
% 
%----------------------------------------------------------------


\chapter{rSlidy}

\label{chap:rslidy}

\textit{rSlidy} is a pure HTML, CSS and JavaScript solution for slide 
presentations in ongoing development at TUG. The project seems to have been 
started by Markus Schofnegger according to the author lines in the core 
JavaScript file. It already had many upgrades in form of projects before it 
came to us for the next upgrade. The average user does not need much more then 
basic HTML coding knowledge to use it, since every presentation is composed in 
a single HTML file with references to the supporting JavaScript core and CSS 
files. The slides themselves are defined as divs with slide class, with the 
content being encapsulate in standard HTML tags, chosen by the user for uniform 
design over the slides. Custom CSS and JavaScript code can be added after the 
neaded resource files.

\begin{minipage}{\linewidth}
\begin{lstlisting}[
language=CSS,
label=list:rslidySetup,
caption={[Connecting rSlidy Resources]%
	By adding the following lines to the head of the HTML slide file, the 
slideshow is initialize on load event. 
}
]
<link rel="stylesheet" href="css/reset.css"/>
<link rel="stylesheet" href="css/normalise.css"/>
<link rel="stylesheet" href="css/rslidy.css" />
<link rel="stylesheet" href="css/slides-default.css"/>

<script src="js/rslidy.js"> </script>
\end{lstlisting}
\end{minipage}

After observing the faults of the project, the goal became to create more user 
friendly plug \& play presentation functions, mainly by animation integration. 
The upgraded rslidy version setup is the same by loading the needed JS and CSS 
in the presentation file, however an addition rslidy-animation.css is needed, 
since it defines the newly added animations, that had no previous component. 
While developing solutions we also found some nice 3rd party components, that 
are presented in enhanced version of our solution. This is also the reason for 
providing 2 solution versions, one pure \textit{rSlidy} solution and an 
enhanced solution.

% section  (end)